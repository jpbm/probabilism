% \chapauthor{J. P. Balthasar Mueller}
\chapter{Linear Algebra and Multivariable Calculus}

\begin{multicols}{2}[\subsubsection*{Contents of this chapter}]
   \printcontents{}{1}{\setcounter{tocdepth}{2}}
\end{multicols}



\section{Multi-Index Notation}
Multi-index notation makes high-dimensional things faster and easier. A collection is indices is represented by a tuple $\alpha = \left(\alpha_1,\alpha_2,\alpha_3,... \right)$. The absolute value $|\alpha| = \sum_i \alpha$, partial derivatives $\partial^\alpha = \prod \partial^{\alpha_i}$, powers $\mathbf{x}^\alpha = \prod_i x_i^{\alpha_i}$.

\subsection{Example: Multinomial Coefficients}	

Instead of:
\begin{equation}
\sum_{0\leq i_1,i_2,i_3,...,i_k \leq n} {n \choose i_1,i_2,i_3,...,i_k} 
\end{equation}

Write:

\begin{equation}	
\sum_{0\leq |\alpha|\leq n}{n \choose \alpha}
\end{equation}

\subsection{Example: Taylor Expansion}

For a vector valued function $\mathbf{f}: \mathbb{R}^n \rightarrow \mathbb{R}^m$ that is analytical in a neighborhood of the point $\mathbf{a}$:

\begin{equation}
f(\mathbf{x}) = \sum_{|\alpha|\geq0} \frac{(\mathbf{x} - \mathbf{a})^\alpha}{\alpha!}(\partial^\alpha)f
\end{equation}




\section{Linear Systems of Equations}
\label{sec:linearequations}

(Real numbers only this time.)

Linear equations are of the form $Ax = b$ where $A$ is a matrix and $x$ and $b$ are vectors. The rows of $A$ and $b$ form a system of equations that must be simultaneously satisfied by the entries of $x$. If $x,b\in\mathbb{R}^n$, then the solutions to the equation of a single row corresponds to an $n-1$-dimensional hyperplane. If the rows of $A$ are linearly independent, then solutions that simultaneously satisfy the equations in $k$-rows correspond to the $n-k$-dimensional intersection of $k$ $n$-dimensional hyperplanes. The solution of $x$ that satisfies all $n$ equations is a $n-n = 0$-dimensional point, and so $x$ is uniquely determined. If any two rows of $A$ are not linearly independent, then the hyperplanes that correspond to values of $x$ that satisfy them overlap exactly, and their intersection is $n$ dimensional, rather than $n-1$ dimensional. In this case, the value of $x$ that satisfies all rows of $A$ is not narrowed down to a single point. The system of equations is said to be *underdetermined*:. This is equivalently the case when $A$ has $m<n$ rows.  

\subsection{$A\in\mathbb{R}^{n\times n}$ Square Matrices}
If $A\in\mathbb{R}^{nxn}$, then the solution to the system is formally $x = A^{-1}b$, where $A^{-1}$ is the matrix inverse, satisfying $A^{-1}A=I$, where $I$ is the identity matrix. 

Since $Ax=b$ is the same as expressing $b$ in terms of a linear combination of the columns of $A$, the entries of $x$ can be interpreted as the coefficients resulting from the projection of $b$ into the column space of $A$. Therefore, for an orthonormal matrix, the $A^-1$ is simply $A^T$ 

\subsection{$A\in\mathbb{R}^{m\times n}$ Rectangular Matrices, Overdetermined Case}
If $A\in\mathbb{R}^{mxn}$ with $m>n$ rows, then there need not be any point $x\in\mathbb{R}^n$ in which the $m$ hyperplanes all intersect. In that case, the system does not have a solution $x\in\mathbb{R}^n$, and the system is considered *overdetermined*. (The intersection of $m$ distinct hyperplanes in $n$ dimensional space would have negative dimension $(n-m)<0$ if $m>n$, which my feeble brain can't make sense of.)
\\

In the overdetermined case $A^{mxn}$ with $m>n$, the columns of $A$ do not span $\mathbb{R}^m$ and therefore $b\in\mathbb{R}^m$ may have some component $\epsilon$ that lies outside of the column space of $A$. In that case, no linear combination $x$ of the columns of $A$ can express $b$ perfectly, but we might look for approximate solutions $\hat{x}$ so that:

\begin{equation}
A\hat{x} + \epsilon = b
\end{equation}

So that the error $||\epsilon||_{\alpha}$ is minimized. This is the starting point for linear regression from the linear algebra perspective. In practice, the approximation is usually approximated by applying an iterative gradient descent algorithm to minimize the \textit{loss function} $||\epsilon||_{\alpha}$. The choice of metric $||\cdot||_{\alpha}$ is essentially a design choice. For $\alpha=2$, the metric is the $L^2$ norm (cf. section \ref{sec:l2norm}) and an analytic solution exists, named the \textit{normal equations}. The solution minimizes the least squares error and corresponds to the projection of $\mathbf{b}$ into the column space of $a$. The procedure is better known as ordinary least squares regression and therefore I will move a more elaborate discussion to chapter \ref{chap:linearregression}.

\subsection{$A\in\mathbb{R}^{n\times m}$ Rectangular Matrices, Underdetermined Case}
For an underdetermined system with $m<n$, there is either no solution (if the $m$ hyperplanes don't intersect), or there are infinitely many possible solutions that lie on an $n-m$ dimensional hyperplane. One idea is to pick the solution that minimizes $||\hat{x}||_2$ based on the idea that it might generalize better. The least norm solution is $\hat{x} = A^T\left(AAT\right)^{-1}b$, which is the projection of $\vec{0}$ on the solution set.

% Cholesky Decomposition
\section{$\mathbf{A} = \mathbf{L}\mathbf{L}^{\dagger}$ Cholesky Decomposition}
\label{sec:cholesky}

The Cholesky Decomposition exists when a matrix is hermitian and positive-definite. It expresses the matrix $\mathbf{A}$ as:

\begin{equation}
\mathbf{A} = \mathbf{L}\mathbf{L^\dagger}
\end{equation}

Where $\mathbf{L}$ is a lower-triangular matrix with positive, real diagonal entries. When $\mathbf{A}$ is real, then so is $\mathbf{L}$. The Cholesky decomposition enables fast solution of a linear system, but it can also be used to create correlated random variables in Monte Carlo simulations. 

\subsection{Creating Correlated Random Variables}
Let $\mathbf{u}_t$ be a vector of uncorrelated samples with mean 0 and 	standard deviation 1. If the covariance matrix of the system to be simulated is  $\mathbf{\Sigma}$ with Cholesky decomposition $\mathbf{\Sigma} = \mathbf{LL}^\dagger$, then the vector $\mathbf{v}_t = \mathbf{Lu}_t$ has the desired covariance.

\begin{figure}
\centering
\includegraphics[scale=0.5]{cholesky1.png}
\includegraphics[scale=0.5]{cholesky2.png}
\caption{Creating correlated random variables from uncorrelated random variables using the Cholesky decomposition of the covariance matrix. The 5 uncorrelated random variables are sampled from a standard normal distribution. It is difficult to see a difference between the correlated and uncorrelated random walks.}
\end{figure}



\section{Generalized Eigenvectors}

\section{$\mathbf{A} = \mathbf{V\Lambda V^{-1}}$ Spectral Theorems, Diagonalization}
\label{sec:diagonalization}

Spectral theorems deal with diagonalizable linear operators. 
\\

A diagonalization of a matrix $\mathbf{A}$ is always possible when a matrix is square, and refers to a decomposition of the matrix into the matrix of eigenvectors $\mathbf{V}$ and eigenvalues $\mathbf{\Lambda}$ as 

\begin{equation}
\mathbf{A} = \mathbf{V\Lambda V^{-1}} 
\end{equation}

\subsection{$\mathbf{A} = \mathbf{V\Lambda V}^T$ Eigendecomposition of Symmetric Matrices}
A hermitian matrix $\mathbf{A}$ has orthogonal eigenvectors, which means that $\mathbf{V}$ is unitary, meaning that $\mathbf{V}^{-1} = \mathbf{V}^{\dagger}$. In that case, the diagonalization is:

\begin{equation}
\mathbf{A} = \mathbf{V\Lambda V}^{\dagger} = \left[\begin{array}{cccc}
\vrule&\vrule&\hdots&\vrule\\
v_1&v_2&\ddots&v_n\\
\vrule&\vrule&\hdots&\vrule
\end{array}\right]\left[\begin{array}{cccc}
\lambda_1&0&\hdots&0\\ 
0&\lambda_2&\hdots&0\\
\vdots&\vdots&\ddots&\vdots\\
0&\hdots&\hdots&\lambda_n
\end{array}\right]
\left[
\begin{array}{ccc}
\rule[.5ex]{3.5em}{0.4pt}&v_1^{\dagger}&\rule[.5ex]{3.5em}{0.4pt}\\
\rule[.5ex]{3.5em}{0.4pt}&v_2^{\dagger}&\rule[.5ex]{3.5em}{0.4pt}\\
\vdots&\ddots&\vdots\\
\rule[.5ex]{3.5em}{0.4pt}&v_n^{\dagger}&\rule[.5ex]{3.5em}{0.4pt}\\
\end{array}
\right]
\end{equation}

Which is the same as saying that all hermitian matrices are \textit{similar} to a diagonal matrix (cf. section \ref{sec:similiarity}, two matrices $\mathbf{A}$ and $\mathbf{B}$ are similar if they are transmutable using unitary transformations as $\mathbf{A} = \mathbf{UBU^{\dagger}}$). The diagonal representation also shows that in order for $\mathbf{A}$ to satisfy the hermitian property $\mathbf{A}^{\dagger}=\mathbf{A}$ its eigenvalues $\lambda_i$ must be real. Further, it means that if $\mathbf{A}$ is hermitian, $\mathbf{A}$ can be written in terms of projections on the eigenvectors:

\begin{equation}
\mathbf{A} = \sum_i \lambda_i (v_i \otimes v_i)
\end{equation}

Where $\otimes$ is the (complex) outer product $v_i\otimes v_i = v_i v_i^{\dagger}$. I have seen the existence of this representation of a hermitian matrix be described as synonmous with \textit{spectral theorem}. Since the eigenvectors $\mathbf{V}$ are an orthonormal basis, $\sum_i v_i \otimes v_i = \mathbb{I}$.


\subsection{$\mathbf{H} = \mathbf{U\Lambda U}^T$ Eigendecomposition of Hermitian Matrices}

Similarly, a hermitian matrix $\mathbf{H} \in \mathbb{C}^{n\times n}$ (the complex equivalent to a symmetric matrix) has real eigenvalues and the matrix of eigenvectors is unitary, so that $\mathbf{H} = \mathbf{U\Lambda U}^T$.


\subsection{Eigenvalue Sensitivity and Accuracy}

\subsubsection{General Case}

In general, the values of the eigenvalues of a square matrix $\mathbf{A}$ may vary wildly under a slight cange of $\mathbf{A} \rightarrow \mathbf{A}+\delta\mathbf{A}$. The sensitivity of the eigenvalues to a change in $\mathbf{A}$ can be investigated using matrix norms \cite{mathworkseig}. Let $||\cdot||$ denote a submultiplicative matrix norm, then:

\begin{equation}
\begin{array}{rl}
\Lambda + \delta\Lambda &= \mathbf{X^{-1}}\left( \mathbf{A} + \delta\mathbf{A} \right)\mathbf{X} \\
\delta\Lambda &= \mathbf{X^{-1}} \delta \mathbf{A} \mathbf{X}\\
||\delta\Lambda || &= ||\mathbf{X^{-1}} \delta \mathbf{A} \mathbf{X} || \leq  ||\mathbf{X^{-1}}|| ||\mathbf{X}|| ||\delta\mathbf{A}||
\end{array}
\end{equation}

When $||\cdot||$ is chosen to be the operator norm with respect to $L^2$, $||\cdot||_{(2)}||$, then $||\mathbf{X^{-1}}|| = \sigma_1$ and $||\mathbf{X}|| = \frac{1}{\sigma_n}$ where $\sigma_1$ and $\sigma_2$ are the square roots of the largest and the smallest eigenvalue of $\mathbf{X^{\dagger}}\mathbf{X}$ respectively (cf. section \ref{sec:norms} on matrix norms). In that case, the sensitivity of the eigenvalues to a change in $\mathbf{A}$ is:

\begin{equation}
||\delta\mathbf{\Lambda}||_{(2)} \leq \frac{\sigma_1}{\sigma_n} ||\delta\mathbf{A}||_{(2)} = \kappa(\mathbf{X})||\delta\mathbf{A}||_{(2)}
\end{equation}

Where $\kappa(\mathbf{X})$ is the conditioning number of the matrix $\mathbf{X}$. Upper bounds on the error on individual eigenvalues can also be derived quite easily, which is shown in \ref{mathworkseig} pp 10-12.

\subsubsection{Hermitian Matrices}
For hermitian (or orthogonal) matrices, the conditioning number for the individual eigenvalues  $\kappa(\lambda_i,\mathbf{H}) = 1$, so that the error on an individual eigenvalue $||\lambda_i||_{(2)} \leq \kappa(\lambda_i,\mathbf{H}) ||\mathbf{H}||_{(2)} = 1\times ||\mathbf{H}||_{(2)}$. 
\section{$\mathbf{A} = \mathbf{U\Sigma V}^{\dagger}$ Singular Value Decomposition}
\label{sec:svd}

The Singular Value Decomposition (SVD) exists for \underline{any} matrix $\mathbf{A}\in\mathbb{C}^{m\times n}$, and is a closely related alternative to the eigendecomposition (cf. section \ref{sec:diagonalization}) that works for non-square matrices. The decomposition comes up incessantly in the context of data analysis. In general, the decomposition has the form:

\begin{equation}
\mathbf{A} = \mathbf{U\Sigma V^{\dagger}}
\end{equation}

Where $\mathbf{U^{\dagger}}\mathbf{U} = \mathbf{I}$, $\mathbf{V}$ is unitary, and $\Sigma$ is a diagonal matrix with real and positive entries $\sigma_i^2$ along the diagonal, so that $\sigma_1 \geq \sigma^2_2 \geq ...\geq \sigma^2_n$. $\mathbf{U}$ is the matrix of left singular vectors, which are the eigenvectors of $\mathbf{A A^{\dagger}}$. $\mathbf{V}$ is the matrix of right singular vectors, which are the eigenvectors of $\mathbf{A^{\dagger} A}$. The singular values are the square roots of the eigenvalues of $\mathbf{A^{\dagger}A}$ or, equivalently, $\mathbf{AA^{\dagger}}$. If the $\mathbf{A}$ is square and symmetric ($\mathbf{A}=\mathbf{A^T}$, cf. section \ref{sec:hermitian}), then the singular values are simply the absolute values of the eigenvalues of $\mathbf{A}$. 
\\


\subsection{Full and Economy SVDs}
While these properties are always true, unfortunately people use a range of conventions when it comes to the size of $\mathbf{U}$, $\mathbf{\Sigma}$ and $\mathbf{V}$. \\

The first convention is for $\mathbf{U}$ and $\mathbf{V}$ to be square, in which case they contain the full set of left and right singular vectors, and $\Sigma$ has dimension $m\times n$, with $0$ entries in rows $i>n$. This is known as the \textit{full SVD}.\\

\citeasnoun{friedman2001elements} uses the the convention where $\Sigma$ is square, i.e. $\mathbf{U}:\ m\times n$, $\mathbf{\Sigma}:\ n\times n$ and $\mathbf{V}:\ n\times n$. This means that $U$ does not contain the full set of $m$ left singular vectors. Note that, in this case, $\mathbf{U^{\dagger}}\mathbf{U} = \mathbf{I}$ but $\mathbf{U}\mathbf{U^{\dagger}} \neq \mathbf{I}$. \possessivecite{friedman2001elements} convention is of advantage in the context of data analysis, where the data matrix tends to be "tall and skinny" (i.e. $m>>n$), and only the first $n$ left singular vectors are relevant. Also, a letting $\Sigma$ be a square matrix significantly simplifies calculations. \possessivecite{friedman2001elements} is known as the \textit{economy SVD}.\\

The two different layouts are illustrated in Figure \ref{fig:full_vs_economy_svd}, which I brazenly copied from \citeasnoun{mathworkseigs}.

\begin{figure}
\centering
\includegraphics[scale=0.5]{full_vs_economy_svd.png}
\caption{Dimensions for Full vs. Economy SVDs}
\label{fig:full_vs_economy_svd}
\end{figure}

\subsection{Matrix Approximation}
The SVD 
\section{Types of Transformations}

\subsection{Affine Transformations}

Affine transformations are the combination of a linear map and a translation, which has the form $f(\mathbf{x}) = \mathbf{A}\mathbf{x} + \mathbf{b}$. 

\begin{equation}
f: V \rightarrow W
\end{equation}

Where $V$ and $W$ are vector spaces. Affine transformations can be expresses as matrices by adding an entry with a constant to the vectors that describe a point in space. For example, for $\mathbf{x} \in \mathbb{R}^n$,  the affine transform $f(\mathbf{x}) = \mathbf{A}\mathbf{x} + \mathbf{b}$ with $A\in\mathbb{R}^{n,n}$ and $x,b \in \mathbb{R}^{n}$ can be expressed as the product of a rectangular matrix $\mathbf{M}$ and a vector $\mathbf{c}$ as:

\begin{equation}
\mathbf{A}\mathbf{x} + \mathbf{b} = \underbrace{\left[\begin{array}{c|c} \mathbf{A} & \mathbf{b} \end{array}\right]}_{\mathbf{M}} \underbrace{\left[\begin{array}{c} \mathbf{x} \\ 1\end{array} \right]}_{\mathbf{c}}
\end{equation}

Where $\mathbf{c}^T = \left[x_1,x_2,x_3,...,x_n,1\right]$ and $\mathbf{M} \in \mathbb{R}^{n,n+1}$.


\subsection{Unitary Transformations}
Unitary transformations are transformations that preserve the inner product, i.e. $\hat{U}x \cdot \hat{U}y = x \cdot y$. As linear transformations, they are represented by unitary matrices (cf. section \ref{sec:unitary}). Unitary transformations include translations, reflections and rotations. 


\subsection{Multilinear Maps}

A multilinear map acts on several vectors in a way that is linear in each of its arguments. A $k$-linear map acts on $k$ vectors, where $k=2$ are bilinear maps and $k=1$ are linear maps.

\begin{equation}	
f: V_1 \times V_2 \times ... \times V_n \rightarrow W
\end{equation}

Where $V_1, V_2, ... , V_n$ and $W$ are vector spaces. An example would be the addition or subtraction of two or more vectors.

\subsection{Multilinear Forms}
Multilinear forms are multilinear maps that have a scalar output. An example is the dot product between two vectors, or summing over the elements of one or more vectors.

\begin{equation}
f: V_1 \times V_2 \times ... \times V_n \rightarrow K
\end{equation}

Where $V_1, V_2, ... , V_n$ and $K$ is a scalar field.


\section{Types of Matrices}

\subsection{$\mathbf{A}^{dagger} = \mathbf{A}$ Hermitian}
Hermitian matrices are matrices that are equal to their complex transpose. That is:

\begin{equation}
{\mathbf{A}^{*}}^T = \mathbf{A}^\dagger = \mathbf{A} 
\end{equation}


Hermitian matrices are like a generalization of symmetric matrices to include complex numbers. 

\subsubsection{Properties}
There are many properties. 
\begin{itemize}
\item By definition: $\mathbf{A} = \mathbf{A}^\dagger$
\item Diagonal Entries are all real, since $a_{i,i} = a_{i,i}^*$
\item Inverse is also hermitian:  $\mathbf{A}^{-1} ={ \mathbf{A}^{-1}}^\dagger$
\item Diagonalizable with real eigenvalues and orthogonal eigenvectors $\in \mathbb{C}^n$.
\end{itemize}


\subsection{$\mathbf{A}^{\datter} = -\mathbf{A}$ Skew Hermitian}
Skew Hermitian matrices that are equal to the negative of their complex transpose. That is:

\begin{equation}
{\mathbf{A}^{*}}^T = \mathbf{A}^\dagger = -\mathbf{A} 
\end{equation}



\subsection{Triangular}
A lower triangular matrix is a matrix that has all-zero entries above the diagonal.

\begin{equation}
\mathbf{L} = \left[\begin{array}{cccccc} l_{1,1}&&&&&0\\l_{2,1}&l_{2,2}&&&&\\l_{3,1}&l_{3,2}&\ddots&&&\\  \vdots&\vdots&\ddots&\ddots&&\\ \vdots&\vdots&&\ddots&\ddots&\\  l_{n,1}&l_{n,2}&\hdots&\hdots&l_{n,n-1}&l_{n,n}\end{array}\right]
\end{equation}

Upper triangular matrices are matrices that have all-zero entries below the diagonal.


\subsection{$\mathbf{U}^{\dagger}\mathbf{U} = \mathbf{I}$ Unitary}
Unitary matrices satisfy $\mathbf{U}^{\dagger}\mathbf{U} = \mathbf{UU}^{\dagger}=\mathbf{I}$, and they have $det(\mathbf{U}) = 1$. They are diagonalizable and can be expressed as $e^{i\mathbf{H}}$ where $\mathbf{H}$ is a Hermitian matrix. Rotation matrices in complex space are Unitary. For real matrices, unitary matrices are the same as orthogonal matrices.

\subsection{$\mathbf{A}^{T}\mathbf{A} = \mathbf{I}$ Orthogonal} 
Orthogonal matrices satisfy $\mathbf{A}^{-1} = \mathbf{A}^T$. The rows (and columns) of $\mathbf{A}$ are an orthonormal basis.



\section{Matrix Norms}

Matrix norms are functions $||\cdot|| K^{m\times n} \rightarrow \mathbb{R}$ where $K$ is a field of real or complex numbers. They satisfy:

\begin{itemize}
\item $||\alpha A|| = |a| ||A||$ (absolutely homogenous)
\item $||A+B|| \leq ||A|| + ||B||$ (triangle inequality)
\item $||A||\geq 0$ (positive valued)
\item $||A||=0 \implies A_{n,m}=0$ (definiteness)
\end{itemize}

A norm is submultiplicative if it satisfied $||AB||\leq||A||||B||$.

\subsection{Operator Norm}
Operator norms 


\subsection{Frobenius Norm}


\subsection{$L_{2,1}$ and $L_{p,q}$ Norms}

\section{Taking Derivatives}

A concise resource for all this is \citeasnoun{barnesmatrixdiff}. Below are some of the most frequent and useful identities.  

\begin{equation}
\begin{array}{l}
\frac{\partial }{\partial \mathbf{x}} \left(\mathbf{u}^T\mathbf{x}\right) = \left[\frac{\partial }{\partial x_1}\left(\sum_i u_i x_i\right),...,\frac{\partial }{\partial x_n}\left(\sum_i u_i x_i\right)\right] = \mathbf{u}^T\\
\\
\frac{\partial }{\partial \mathbf{x}} \left(\mathbf{x}^T\mathbf{u}\right) = \left[\frac{\partial }{\partial x_1}\left(\sum_i u_i x_i\right),...,\frac{\partial }{\partial x_n}\left(\sum_i u_i x_i\right)\right] = \mathbf{u}^T\\
\\
\frac{\partial }{\partial \mathbf{x}} \left(\mathbf{x}^T\mathbf{x}\right) = \left[\frac{\partial }{\partial x_1}\left(\sum_i x_i^2\right),...,\frac{\partial }{\partial x_n}\left(\sum_i x_i^2\right)\right] = 2\mathbf{x}^T\\
\\
\frac{\partial }{\partial \mathbf{x}} \left(\mathbf{Ax}\right) = \left[
\begin{array}{ccc} 
\underbrace{\frac{\partial }{\partial x_1}\left(\sum_i A_{1i} x_i\right)}_{A_{11}} &...& \underbrace{\frac{\partial }{\partial x_n}\left(\sum_i A_{1i} x_i\right)}_{A_1n}\\
\vdots&\vdots&\vdots\\
\underbrace{\frac{\partial }{\partial x_1}\left(\sum_i A_{ni} x_i\right)}_{A_{n1}} &...& \underbrace{\frac{\partial }{\partial x_n}\left(\sum_i A_{ni} x_i\right)}_{A_{nn}}\\
\end{array}\right] = \mathbf{A}\\
\end{array}
\end{equation}

The derivative of a matrix after a vector is known as the Jacobian. The Jacobian of some function $f: \mathbb{R}^n \rightarrow \mathbb{R}^m$ is:

\begin{equation}
\frac{\partial  f(\mathbf{x}}{\partial \mathbf{x}}=\left[\begin{array}{ccc}
\frac{\partial  f_1}{\partial  x_1} & \hdots & \frac{\partial  f_1}{\partial  x_n} \\
\vdots & \vdots & \vdots \\
\frac{\partial  f_m}{\partial  x_1} & \hdots & \frac{\partial  f_m}{\partial  x_n} \\
\end{array}\right]
\end{equation}




\chapauthor{}
