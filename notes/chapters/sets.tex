\chapter{Sets and Measure Theory	}

\begin{multicols}{2}[\subsubsection*{Contents of this chapter}]
   \printcontents{}{1}{\setcounter{tocdepth}{2}}
\end{multicols}



Sets are a term to describe collections of things. The things could be countable objects, such as the integers between $1$ and $10$, or contain a continuum, for example all the real numbers between $1$ and $10$. Unless otherwise specified, the elements of a set are assumed to be distinct and as not having an internal order. That is, the set of letters in "Mississippi" is $\{M,i,s,p \} = \{i,p,M,s 	\} = ... $ and so on. A concise resource for notation is \citeasnoun{stanfordsetnotation}. 


% representation
\section{Representation}

\begin{tabular}{ll}
Statement Form & $\{\mathrm{integers\ between\ 1\ and\ 5}\}$\\
Roster Form & $\{1,2,3,4,5\}$ \\
Setbuilder Form & $\{x|x\in\mathbb{N}, 1\leq x \leq 5 \} $
\end{tabular}


% Set Properties, Types of Sets
\section{Set Properties and Types of Sets}

\subsection{Cardinality}
The cardinality of a set is a measure of the size of a set. For finite sets, the cardinality is simply the number of elements. For example, the set  $A = \{a,b,c,d\}$ has size $|A|=4$.

For infinite sets, this intuition breaks down, though it is still possible to make meaningful comparisons. 

\subsection{$\emptyset$ Empy Set, Null Set}
The empty set is the set with no elements. It is denoted $\{\}$ or $\emptyset$. 

\subsection{Singleton Set}
A singleton set is a set with one element. I.e., $\{a\}, \{b\}$ are singleton subsets of $\{a,b\}$. A singleton has cardinality 1.

\subsection{Infinite Sets}

\subsection{Multisets}
A multiset is a set that may contain an element more than once. I.e. it makes sense to write $A=\{a,a,a,b,b\}$. The number of times an element is included is the \textit{multiplicity}. The multiset $A$ may be defined in terms of a \textit{multiplicity function} $m_A(x)$, which describes the multiplicity of a type of element $x\in U$ where $U$ may be referred to as the \textit{universe} (en lieu of saying "univseral"!). The multiplicity function allows the extension of set characteristics and operations to multisets.

\subsubsection{Support}
The support of a multiset is:

\begin{equation}
Supp(A) = \{ x\in U | 	m_A(x)>0\}
\end{equation} 

Which is the set of distinct elements in $A$. I.e., if $A = \{a,a,a,b,b,c\}$ then $Supp(A) = \{a,b,c\}$.

\subsubsection{Cardinality}

The cardinality is given by:

\begin{equation}
|A| = \sum_{x\in U} m_A(x)
\end{equation}

\subsubsection{$\subseteq$ Inclusion}
The concept of subsets can be extended to multisets as:

\begin{equation}
A \subseteq B
\end{equation}

if 

\begin{equation}
\forall x\in U, \ \ m_A(x) \leq m_B(x)
\end{equation}


\subsubsection{$\bigcap$ Intersection}
The intersection of multisets is sometimes called the \textit{infimum} or \textit{greatest common divisor}. If $A\cap B = C$, then $C$ has multiplicity function:

\begin{equation}
m_C(x) = min\left(m_A(x),m_B(x)\right)
\end{equation}

That is, it is like an elementwise minimum.

\subsubsection{$\bigcup$ Union}
In the context of multisets, the term \textit{union} sometimes refers to multiset addition. Otherwise, it should refer to the overlap of two multisets, i.e.:
 
\begin{equation}
m_C(x) = max\left(m_A(x),m_B(x)\right)
\end{equation}

That is, it is like an elementwise maximum.

\subsubsection{$\uplus$ Multiset Addition} 
In contrast to sets in which distinct elements are only contained once, multiset addition makes sense and uses a special symbol "$\uplus$". If $A\uplus B = C$, then $C$ has multiplicity function:

\begin{equation}
m_C(x) = m_A(x) + m_B(x)
\end{equation} 

\subsubsection{Multiset Subtraction}
Multisets can be subtracted, with the condition that the multiplicity can not be less than zero. If $A- B = C$, then $C$ has multiplicity function:

\begin{equation}
m_C(x) = max\left(m_A(x) - m_B(x), 0\right)
\end{equation}

% Powerset
\subsection{Powersets}
The powerset $P(S)$ of a set $S$ is the set of all subsets of $S$, including the empty set and the set itself. That is:

\begin{equation}
P(S) = \{ A : A\subseteq S\}
\end{equation}

My guess is it's called powerset because the number of subsets of S, $|P(S)| = 2^{|S|}$. I have also seen the notation $2^S$ to refer to the space of subsets of $S$.



% Measurable Sets
\subsection{Measurable Sets}
Measurable sets have a way of measuring volume on them in a non-trivial way. That is, for some subset of a measurable set $A_i \subseteq A$, it is possible to define a metric $\mu$ so that $\mu(A_i) \neq0)$. Measurable sets are the elements of a $\sigma$-algebra. 

% universal set
\subsection{Universal Set}
The \textit{universal set} is understood to refer to something that is not allowed in the context of Russell's Paradox. It is nevertheless useful to define a set $S$ so that $S^c = \emptyset$ and I've seen this type of set referred to as \textit{universal set} a couple of times. In the context of probability theory, $S$ may be the whole event space.




%% open sets, closed sets
\subsection{Open Sets, Closed Sets}
Open sets and closed sets are generalizations of open and closed intervals on the real line. Open sets a mentioned plenty in the context of measure theory. A set is open if and only if its complement is closed, and a set is closed if and only if its complement is open. Counterintuitively, a set can be both open and closed at the same time, or it can be neither open nor closed at the same time.

\subsubsection{Interior Points}
An interior point of a set $A$ is any point $X$ so that for some $\epsilon > 0$, the open interval $X-\epsilon,X+\epsilon \subseteq A$ is contained within $A$. One might imagine an interior point as a point that has a neighborhood (a "ball") that is included in $A$. 

\subsubsection{Accumulation Points} 
In contrast, accumulation points do not have a neighborhood contained in $A$. Formally, for some $\epsilon>0$, $(X-\epsilon,X+\epsilon)\cap (A\setminus \{X\}) \neq \emptyset$, which is only possible for a point that is on the very boundary of $A$.

\subsubsection{Open Sets}
Unsurprisingly, an open set is a set that only has interior points, i.e. $\{X: X\in A, \exists \epsilon > 0 \mathrm{\ s.\ th.\ } X-\epsilon,X+\epsilon \subseteq A  \}$.

\subsubsection{Closed Sets}
Closed sets contain all of their accumulation points. That is, the boundary is included. Points might also be standing alone, for example:

\begin{equation}
[2,4]\cup\{1\}
\end{equation}

Is a closed set.

\subsubsection{Both Open and Closed, Neither Open Nor Closed}
The set of real numbers $\mathbb{R}$ is both open and closed. Any point on the real line has a neighborhood in $\mathbb{R}$, so it is open. On the other hand, the accumulation points of $\mathbb{R}$ are $\mathbb{R}$, so it is closed. An interval that has some but not all accumulation points is neither open nor closed. For example $[1,2)$.	

%% Image
\subsection{Image}
\label{sec:image}
The image are the elements of the codomain that a given subset of the domain is mapped to. For example, given a function $f:\mathbb{R} \rightarrow \mathbb{R}: f(x) = x^2$, the image of $\{-3,-2,2,3\}$ in the domain are the elements $\{4,9\}$ of the codomain.

%% Preimage 
\subsection{Preimage, Inverse Image}
\label{sec:preimage}
The preimage are elements of the domain that are mapped to a given subset of the codomain. For example, given a function $f:\mathbb{R} \rightarrow \mathbb{R}: f(x) = x^2$, the preimage of the elements $\{4,9\}$ of the codomain are the elements $\{-3,-2,2,3\}$ of the domain..


%% Convex Sets
\subsection{Convex Sets}
Convex sets are subsets of vector spaces in which any point along the line connecting to points within the set is contained within the set. A disk in $\mathbb{R}^2$ is a convex set. A crescent in $\mathbb{R}^2$ is not a convex set. The boundary of a convex set is a convex function. For example, a parabola $f(x) = x^2$ with $x\in\mathbb{R}$ is a convex function, and the area above the parabola, called the epigraph, is a convex subset of $\mathbb{R}$. Convex optimization deals with the optimization of convex functions over convex sets. The intersection of convex sets is always convex, but the union of convex sets is only convex under certain conditions. 

More generally, if $S$ is a convex set, then affine combinations of the elements $\mathbf{x}_i  \in S$ of the form:

\begin{equation}
\sum_i \mathbf{x}_i \lambda_i 
\label{eq:convexsets}
\end{equation}  

With $\sum_i \lambda_i = 1$ and $\lambda_i\geq 0$ are also contained in $S$. Rather than just the points along the line between two points, these are the essentially the weighted averages of multiple points in the set.  

\subsubsection{Example: Discrete Probability Distributions}
The set of discrete probability distributions $P = \{ \mathbf{p}=(p_1,p_2,...): ||\mathbf{p}||_1 = 1, p_i \geq 0\}$ is a convex set. Property \ref{eq:convexsets} means that a weighted average over members of $P$ is also a member of $P$, so long as the weights satisfy $\sum_i \lambda_i = 1$. In general, this guarantees that marginalization results in a probability. I.e. if $p(X|Y) = \mathbf{p}(X|Y) \in P$ and $p(Y) = \mathbf{p}(Y) \in P$, then $\mathbf{p}(X) = \sum_i p_i(X|Y) p_i(Y) \in P$. 

%% Choice sets, Transverse Sets, Cross-sections
\subsection{Choice Sets, Transversal Sets, Cross-Sections}
\label{sec:choicesets}
Choice sets, transversal sets or cross-sections are sets that are assembled by picking exactly one element from each member of a family of disjoint sets. The axiom of choice (cf. section \ref{sec:axiomofchoice}) states that such a set can always be formed.

\subsubsection{Example: Integers}
Consider the partition of the integers between $1$ and $40$ into a family of disjoint sets that each contain $10$ elements: 

\begin{equation}
S = \{ \{1,...,10\}, \{11,...,20\}, \{21,...,30\},\{31,...,40\} \}
\end{equation}

Then a transversal set $A$ could for example be formed by picking the smallest number in each of the subsets. 

\begin{equation}
A = \{1,11,21,31\}
\end{equation}

Though we might have picked any other choice function.  


% Set operations
\section{Set Operations}

\subsection{$A^c$ Complement}
Given a subset $A \subseteq X$, the complement of the subset $A^c$ refers to everything that is not contained in $A$, i.e. $A^c = X\setminus A$. 


\subsection{$\bigcup$ Union}
The union of sets is the set that contains all of their elements, counting all elements only once. The cardinality of the union is calculated using the important inclusion exclusion principle (\ref{sec:inclusionexclusion}). In terms of logic, think $A\cup B$ is "A or B (or both)".

\subsection{$\bigcap$ Intersection}
The intersection of sets is the elements shared by all sets. $A\cap B$ is "A and B".

% Disjoint Union
\subsection{$\bigsqcup$ Disjoint Union, Discriminated Union}
The disjoint union, or discriminated union, of sets is the union formed in a way in which the information about which subset an element belonged to is preserved. One way to write this is to include a subset index with each element, i.e.:

\begin{equation}
\bigsqcup_{i\in\{i\}}A_i = \bigcup_{i\in\{i\}} \{(x,i): x\in A_i\}
\end{equation}

For example, for the two sets $A=\{1,2,3,4\}$ and $B=\{1,2,3\}$, the disjoint union is:

\begin{equation}
A\sqcup B = \{(1,A),(2,A),(3,A),(4,A),(1,B),(2,B),(3,B)\}
\end{equation}

The cardinality of the disjoint union is simply the sum:

\begin{equation}
\left|\bigsqcup_{i\in\{i\}}A_i\right| = \sum_{i\in\{i\}}|A_i
\end{equation}

I have also seen the disjoint union be used in the context of forming the union of disjoint sets, perhaps to stress that the sets are disjoint and the cardinality can be calculated through simple summation as above. For example, in dividing up the interval: $[0,2]=[0,1)\sqcup[1,2]$.


% Bijection Principle
\section{Bijection Principle}
The bijection principle states that when a bijection exists between the elements of two sets, then those sets have the same size. This principle is very useful in combinatorics, and it allows for meaningful comparisons of infinite sets.


% DeMorgan
\section{DeMorgan's Rules}
\label{sec:demorgan}
DeMorgan's Rules relate the complement of the union to the intersection of the complements, and the complement of the intersection to the union of the complements.

\begin{equation}
\left(\bigcup_{i\in\{i\}}A_i\right)^c = \bigcap_{i\in\{i\}}A^c_i
\end{equation}

\begin{equation}
\left(\bigcap_{i\in\{i\}}A_i\right)^c = \bigcup_{i\in\{i\}}A^c_i
\end{equation}


%Inclusion Exclusion
\section{Inclusion - Exclusion Principle}
\label{sec:inclusionexclusion}

The inclusion-exclusion principle is used to calculate the size of the union of sets. This requires counting each region of some complicated overlapping Venn diagram exactly once, which, in turn requires accounting for overcounting wherever sets overlap. Let $\{A_i | i\in \{i\}_n \}$ be a collection of $n$ overlapping sets indexed by $i\in \{i\}_n$, then the inclusion-exclusion principle is given by:

\begin{equation}
\left|\bigcup_{i\in\{i\}_n} A_i\right| = \sum^n_{k=1} (-1)^{k-1} \sum_{\{j\}_k \subseteq \{i\}_n} \left|\bigcap_{j\in\{j\}_k} A_j\right|
\end{equation}


Where the sum over $\{j\}_k \subseteq \{i\}_n$ is over all $k$-element subsets of $\{i\}_n$.

\subsection{Example: n=2 Sets and n=3 Sets}

\subparagraph{n=2}
\begin{equation}
\begin{array}{rl}
\left|\bigcup_{i\in\{1,2\}}A_i\right| =&  \sum^2_{k=1} (-1)^{k-1} \sum_{\{j\}_k \subseteq \{1,2\}} \left|\bigcap_{j\in\{j\}_k} A_j\right|\\
=&(-1)^{0}\left(|A_1| + |A_2|\right) \\
&+ (-1)^{1}\left(|A_1 \cap A_2| \right)
\end{array}
\end{equation}


\subparagraph{n=3}
\begin{equation}
\begin{array}{rl}
\left|\bigcup_{i\in\{1,2,3\}}A_i\right| =&  \sum^3_{k=1} (-1)^{k-1} \sum_{\{j\}_k \subseteq \{1,2,3\}} \left|\bigcap_{j\in\{j\}_k} A_j\right|\\
=&(-1)^{0}\left(|A_1| + |A_2| + |A_3|\right) \\
&+ (-1)^{1}\left(|A_1 \cap A_2|  + |A_1 \cap A_3| + |A_2 \cap A_3|  \right) \\ 
&+ (-1)^{2}\left(|A_1 \cap A_2 \cap A_3|   \right)
\end{array}
\end{equation}

\subsubsection{Example: Counting Integers}

How many integers are there between 1 and 100 that are neither divisible by 3,5 nor 7?

Let $S$ be the set of all integers between 1 and 100. The size of the set is $|S| = 100$. The subset of $S$ that is numbers divisible by 3 is $A_3 \subseteq S$ with $|A_3| = 33$ because $100/3 = 33.\overline{333}$. Similarly, $|A_5| = 20$ and $|A_7| = 14$.  The set of integers that is not divisible by 3, 5 or 7 is:

\begin{equation}
S \setminus \bigcup_{i\in\{3,5,7\}} A_i
\end{equation}

So that the sought after quantity is :

\begin{equation}
\begin{array}{rl}
\left|S \setminus \bigcup_{i\in\{3,5,7\}} A_i\right| =& |S| - \left[ |A_3| + |A_5| + |A_7| \right.\\
& \left. - |A_3 \cap A_5| - |A_3 \cap A_7| - |A_5\cap A_7| + |A_3\cap A_5\cap A_7|\right]
\end{array}
\end{equation}

The size of the intersection $|A_3\cap A_5| = 6$ because 100 is 6 times divisible by $3\times 5 = 15$. Similarly, $|A_3 \cap A_7| =  4$, $|A_5 \cap A_7| =  2$ and $|A_3\cap A_5 \cap A_7| =  0|$. Hence:

\begin{equation}
\left|S \setminus \bigcup_{i\in\{3,5,7\}} A_i\right| = 100 - 33 - 20 - 14 + 6 + 4 + 2 - 0 = 45
\end{equation}

There are 45 integers between 1 and 100 that are not divisible by 3, 5 or 7.


\section{Axiom of Choice}
\label{sec:axiomofchoice}
The axiom of choice simply states that, given a collection of nonempty, mutually disjoint sets, it is possible to assemble a \textit{transversal} or \textit{choice} set that consists of exactly one element from each of the sets in the collection. For example, consider the students in 1st, 2nd, 3rd.. etc grades at a school to be a collection of nonempty, mutually disjoint sets. Then the axiom of choice says that it is possible to assemble a subset of students with exactly one student from each grade. 


In terms of functions, one might think of defining a \textit{choice function} on the family of mutually disjoint sets, which selects members from the collection of sets and adds them to the \textit{choice} set. According to the axiom of choice, a choice function can be defined for any collection of nonempty, mutually disjoint subsets. This implies that all surjective functions have a right inverse. That is, for any surjective function $f:X\rightarrow Y$, there exists a function $g:Y\rightarrow X$ so that $f(g(y)) = y$.   

The axiom of choice turns out to be associated with famous names and deep consequences \cite{stanfordaxiomofchoice}.

\subsection{Example: Pairs of Real Numbers, Right Inverse}
The collection of rank-2 sets of pairs of real numbers $A= \{ A_x = \{x,-x\} : x\in \mathbb{R}^{+}\}$ are a collection of mutually disjoint subsets. A transversal set $B$ may be assembled by choosing the largest element of the tuple: $B = \{y : y max(A_x), A_x \in A \}$. An equivalent surjection $f:A\rightarrow B$ is $f(A_x) = x$. There is a right-inverse $g:B\rightarrow A$ which is $g(x) = (x,-x)$ so that $f(g(x)) = x$.







\section{$\sigma$-Algebras}

\subsection{Definition}
Given a set $X$, a $\sigma$-Algebra $\mathscr{A}$ is a collection of subsets of a given set $X$, which has to satisfy the conditions:


\begin{itemize}
\item $\emptyset, X \in \mathscr{A}$
\item If $A\in\mathscr{A}$, then $A^c := X\setminus A \in \mathscr{A}$
\item $\mathscr{A}$ has (possibly infinitely many) countable subsets $A_i \in \mathscr{A},\ i \in \mathbb{N}$. Then $\bigcup_{i=1}^{\infty}A_i \in \mathscr{A}$. That is, the $\sigma$-algebra is closed under countable unions of its subsets. 
\end{itemize}

An element of the $\sigma$-Algebra are $\mathscr{A}$-measurable sets, or measurable with respect to the $\sigma$-Algebra $\mathscr{A}$. A $\sigma$-algebra is a subset of the power set of $S$, i.e. $\mathscr{A} \subseteq P(X)$. 

\subsubsection{Example: The Smallest Possible $\sigma$-Algebra}

\begin{equation}
\mathscr{A} = \{\emptyset, X\}
\end{equation}

\subsubsection{Example: The Largest Possible $\sigma$-Algebra} 
The largest possible $\sigma$-Algebra is the power set:
\begin{equation}
\mathscr{A} = P(X)
\end{equation}

However, there are important examples where it is not possible to define a $\sigma$-algebra on the full power set. 


\subsection{Intersection Property}
The intersection of $\sigma$ algebras is also a $\sigma$-algebra. That way, a $\sigma$-algebra with the desired properties can be constructed by creating individual $\sigma$-algebras with the properties in question and forming their intersection.

\begin{equation}
\bigcap_i \mathscr{A} \ \ \mathrm{is\ also\ a\ }\sigma-\mathrm{algebra}
\end{equation}


\subsection{Generated $\sigma$-Algebras}

For a subset of the powerset $\mathscr{M} \subseteq P(X)$, that does not necessarily have to satisfy the properties of the $\sigma$-Algebra, the \textit{smallest} $\sigma$-algebra that contains $\mathscr{M}$ is the $\sigma$-algebra \textit{generated} by $\mathscr{M}$. It can be constructed through the intersection of all $\sigma$-algebras  on $X$ that contain $\mathscr{M}$. 

\begin{equation}
\sigma(\mathscr{M}) = \bigcap_{\mathscr{A}\supseteq\mathscr{M}}\mathscr{A}
\end{equation}

$\sigma(\mathscr{M})$ is the $\sigma$-algebra \textit{generated} by $\mathscr{M}$. 

\subsubsection{Example: Generated $\sigma$-Algebra}

Take $X = \{a,b,c,d\}$, and $\mathscr{M} = \{ \{a\}, \{b\}\}$. Note that $\mathscr{M}$ is not a $\sigma$-algebra. To find the smallest $\sigma$-algebra that contains $\mathscr{M}$, one adds the elements necessary to fulfill the conditions on a $\sigma$-algebra. These are the empty set $\emptyset$ and the full set $X$, the union $\{a,b\}$, and the complements.

\begin{equation}
\sigma(\mathscr{M}) = \left\{ \emptyset, X, \{a\}, \{b\}, \{a,b\}, \{b,c,d\}, \{a,c,d\}, \{c,d\} \right\}
\end{equation}


\subsection{Borel $\sigma$-Algebras ($\mathscr{B}$)	}

Borel $\sigma$-Algebras is the $\sigma$-Algebra generated by \textit{open sets}, for example $\sigma(\mathbb{R}^n)$ or $\sigma(\mathscr{M})$ with $\mathscr{M} = (0,1)$.


\section{Measures, Measurable Spaces and Measure Spaces}
A \textit{measure} is a function that gives a volume measure for subsets of a $\sigma$-algebra, where a $\sigma$-algebra is a special sort of collection of subsets of some set $X$. 
The combination $(X,\mathscr{A})$ of a set $X$ and a $\sigma$-Algebra that is defined on $X$ is called a \textit{measurable space}. A measure is a function defined on the $\sigma$-Algebra and maps to the positive real line (including $\infty$!):

\begin{equation}
\mu: \mathscr{A} \rightarrow [0,\infty)\cup\{\infty\} 
\end{equation}

It has to satisfy the conditions:

\begin{itemize}
\item $\mu(\emptyset) = 0$
\item $\mu(\bigcup_i A_i) = \sum_i \mu(A_i)$ if $A_i \cap A_j = \emptyset$ when $i\neq j$ (additive)
\item $\mu(\bigcup^{\infty}_i A_i) = \sum^{\infty}_i \mu(A_i)$ if $A_i \cap A_j = \emptyset$ when $i\neq j$ ($\sigma$-additive) 
\end{itemize}

Where the infinite sum corresponds to gradually approximating the full volume by taking the union of countably infinitely many subsets. The collection $(X,\mathscr{A},\mu)$ is a \textit{measure space}. 

The reason why the $\sigma$-algebra is included in the definition of a measurable space is because unless a $\sigma$-algebra is defined on a set $X$, there is no guarantee that a measure exists on $X$.

\subsection{Example: Counting Measure}

\begin{equation}
\mu(A) = \left\{\begin{array}{l} |A| \mathrm{\ if\ }A\mathrm{\ has\ finite\ elements}\\ \infty\mathrm{\ else}\end{array}\right.
\end{equation}

\subsection{Example: Dirac Measure}
\begin{equation}
\delta_p(A) = \left\{\begin{array}{l} 1 \mathrm{\ if\ }p\in A\\ 0 \mathrm{\ else} \end{array}\right.
\end{equation}

\subsection{Example: Volume Measure}
For $X=\mathbb{R}^n$, the conventional volume measure satisfies the properties of:

\begin{itemize}
\item $\mu([0,1]^n) = 1$
\item $\mu(x + A) = \mu(A)$ for some $x\in \mathbb{R}^n$ (translation invariance)
\end{itemize}



\section{Length Measure on the Real Line}

This is an important example of where it is not possible to define a measure on the whole powerset. This is why measure theory is founded on $\sigma$-algebras.\\

Take the following measure problem:\\

We search a measure $\mu$ on $P(\mathbb{R})$ with the properties:

\paragraph{}
\begin{enumerate}
\item $\mu([a,b]) = b-a,\ b>a$
\item $\mu(x+A) = \mu(A),\ A\in\mathbb{R},\ x\in\mathbb{R}$	 
\end{enumerate}

The only solution to this problem is the trivial map, $\mu(\mathbb{R}) = 0$. A proof goes as follows:

\paragraph{Claim}
Let $\mu$ be a measure on $P(\mathbb{R})$ with $ \mu ((0,1]) < \infty  $ and (2). The only measure that satisfies this condition is the zero measure, $\mu = 0$, which violates (1).

Take the interval $I:=(0,1]$ with equivalence relation $x\sim y \equiv x-y \in \mathbb{Q}$. That is, $x$ and $y$ are considered equivalent if they differ by a rational number. That is, we define sets of equivalent numbers, equivalence classes  $ [x] := \{ x+r: r \in \mathbb{Q}, x+r \in I \} $. The equivalence classes are a disjoint decomposition (a partition) of the unit interval $I$ in terms of possibly infinite, countable number of elements.

Take a choice set  $A\subseteq I$ that consists of one number from each of the equivalence classes $[x]$ that make up $I$. (cf. sections \ref{sec:choicesets}, \ref{sec:axiomofchoice}). It has the property: 
\begin{itemize}
\item For each $[x]$ there is an $a\in A$ with $a \in [x]$ 
\item For all $a,b \in A$, if $a,b \in [x] \implies a=b$ 
\end{itemize}

Take the translations of a set $A$, $A_n := r_n + A$ where $(r_n)_{n\in\mathbb{N}}$  are an enumeration of $\mathbb{Q}\cap(1,1]$. 

\paragraph{Proof}
The axiom of choice guarantees that it is possible to form a choice set $A$ and the definition of equivalence classes guarantees that the translations of $A_n$ are still choice sets. 

\paragraph{Claim}
$A_n \cap A_m = \empty$ if $n\neq m$.

\paragraph{Proof}
Take $x \in A_n \cap A_m \implies \begin{array}{ll} x = r_n + a, & a_n \in A\\ x = r_m + a_m, & a_m \in A \end{array}$

Then, $r_n + a_n = r_m + a+m  \implies a_n - a_m = r_m - r_n \in \mathbb{Q} \implies a_n \sim a_m$ according to the definition of the equivalence classes. Also, $a_m, a_n \in [a_m] \implies a_n = a_m \implies r_n = r_m \implies n=m$.

\paragraph{Claim}
$(0,1] \subseteq \bigcup_{n\in\mathbb{N}}A_n \subseteq (-1,2]$ 

\paragraph{Proof}
Given $r_n \in \mathbb{Q}\cap (-1,1]$, $-1< r_n \leq 1$ and $a_n \in A,\ 0<a_n\leq 1$ given that $A\subset (0,1]$, $-1 < r_n + a_n \leq 2$. Therefore $A_n \subseteq (-1,2]\ \forall n\in\mathbb{N}$. Therefore the union $(0,1] \subseteq \bigcup_{n\in\mathbb{N}}A_n \subseteq (-1,2]$.


\paragraph{Assume} $\mu$ a measure on $P(\mathbb{R})$ with $\mu((0,1]) < \infty$ and (2). 

By (2):  $\mu(r_n + A) = A$ for all $n\in\mathbb{N}$.

By the property $(0,1] \subseteq \bigcup_{n\in\mathbb{N}}A_n \subseteq (-1,2]$, $\mu((0,1]) \leq \mu(\bigcup_{n\in \mathbb{N}} A_n) \leq \mu((-1,2])$. 
	
Further, $\mu((0,1]) = c < \infty$. 

So we can use the disjoint union to express $\mu((-1,2]) = \mu((-1,0] \sqcup (0,1] \sqcup (1,2]) = 3c$. 

That implies, $c \leq \sum_{n=1}^{\infty}\mu(A_n) \leq 3c \implies c \leq \sum_{n=1}^{\infty}\mu(A) \leq 3c$. Given that the series is infinite, the only way for this to be true is if $\mu(A)=0$. 

\paragraph{Conclusion}
The measure $\mu(A) = 0$, which implies $\mu((0,1]) = 0$. Because of translation invariance and $\sigma$-additivity, $\mu(\mathbb{R}) = \mu(\bigcup_{m\in\mathbb{Z}^+} (m,m+1]) = 0$. That means that the only measure that satisfies the conditions stated in the measure problem assigns $0$ to the length of the whole real line. 


  
\section{Measurable Maps}
\paragraph{Definition} Given two measurable spaces $(\Omega_1,\mathscr{A}_1), (\Omega_2, \mathscr{A}_2)$, a measurable map with respect to $\mathscr{A}_1, \mathscr{A}_2$ is $f:\Omega_1 \rightarrow \Omega_2$ if $f^{-1}(A_2)\in\mathscr{A}_1$ for all $A_2 \in \mathscr{A}_2$. That is, $f$ connects the two $\sigma$-algebras $\mathscr{A}_1$ and $\mathscr{A}_2$ in that the \textit{preimage} (cf. section \ref{sec:preimage}) of an element $A_2 \in \mathscr{A}_2$, $f^{-1}(A_2)$ is an element of $\mathscr{A}_1$. 




\subsection{Example: Characteristic Function, Indicator Function}
Take the measurable spaces $(\Omega,\mathscr{A})$ and $(\mathbb{R},\mathscr{B}(\mathbb{R}))$. 

\begin{equation}
\chi_A: \omega \rightarrow \mathbb{R},\ \ \chi_A(\omega) = \left\{\begin{array}{l} 1,\ \omega\in A\\ 0,\ \omega\notin A \end{array}\right.
\end{equation}

For all measurable $A\in\mathscr{A}$, $\chi_A$ is a measurable map. The four possible preimages are:

\begin{equation}
\begin{array}{l}
\chi_A^{-1}(\empty) = \empty,\ \chi_A^{-1}(\mathbb{R}) = \Omega\\
\chi_A(\{ 1 \}) = A\ \chi_A(\{ 1 \}) = A^c
\end{array}
\end{equation}

Where all of the preimages are contained in $\mathscr{A}_1$.


\subsection{Example: Composition of Measurable Maps}

Take measurable spaces $(\Omega_1,\mathscr{A}_1),(\Omega_2,\mathscr{A}_2),(\Omega_3,\mathscr{A}_3)$,  connected through measurable maps:

\begin{equation}
\begin{array}{l}
f: \Omega_1 \rightarrow \Omega_2 \\
g: \Omega_2 \rightarrow \Omega_3
\end{array}
\end{equation}

Then $f \circ g: \Omega_1 \rightarrow \Omega_3$ is also measurable, because $(g\circ f)^{-1}(A_3) = \underbrace{f^{-1}(\underbrace{g^{-1}(A_3)}_{\in \mathscr{A}_2})}_{\in \mathscr{A}_1}$.	 


\subsection{Example: Sums and Products of Measurable Maps}
Given $(\Omega,\mathscr{A}), (\mathbb{R},\mathscr{B}(\mathbb{R})$, if $f,g: \Omega \rightarrow \mathbb{R}$ 	are measurable maps, then $f+g, f-g, f\times g$ and $|f|$ are also measurable maps. This follows from the property that compositions of measurable maps are also measurable.



\section{Lebesgue Integrals}
\subsection{Lebesgue Integrals for Step Functions}
Take a measure space $(X,\mathscr{A}, \mu)$ and the measurable space $(\mathbb{R},\mathscr{B})$, where $X$ is any set, $\mathscr{A}$ is a special collection of subsets of $X$, $\mu$ is a map $\mu: \mathscr{A} \rightarrow [0,\infty]$, and $\mathscr{B}$ is a Borel $\sigma$-algebra on $\mathbb{R}$. Then let $f:X \rightarrow \mathbb{R}$ be a measurable map, so that $f^{-1}(E) \in \mathscr{A}$ for all $E\subseteq \mathscr{B}(\mathbb{R})$. 

\paragraph{Characteristic Function}
The integral of a characteristic function $\chi_A$ is $I(\chi_A) = \mu(A)$. 

\paragraph{Simple Functions}
Simple functions are, for example, step functions, staircase functions, etc., that can be expressed in terms of a sum of characteristic functions. That is, for $A_1,A_2,...,A_n\in\mathscr{A}$, $c_1,c_2,...,c_n\in\mathbb{R}$:

\begin{equation}
f(x) = \sum_{i=1}^n c_i \chi_{A_i}(x)
\end{equation}

Since characteristic functions are measurable and sums of characteristic functions are measurable, simple functions are measurable. Then the integral would be:

\begin{equation}
I(f) = \sum_{i=1}^n c_i \mu(A_i)
\end{equation}

However, there is a problem with this definition, because of the possibility of having to subtract infinitely large intervals. The options are to restrict simple functions to either: 

\begin{itemize}
\item restrict $A_i$ to be finite size sets.
\item restrict $c_i$ to be positive. 
\end{itemize}

\paragraph{S+}
The set of positive functions $S^+ := \{ f:X\rightarrow\mathbb{R} : f \mathrm{\ simple\ function},\ f\geq0\}$ where $f$ is measurable and has finitely many values (picture a staircase, rather than a smooth curve). For $f\in S^+$, choose representation $f(x) = \sum_{i=1}^n c_i \chi_{A_i}(x)$, $c_i \geq 0$. 



\paragraph{Lebesgue Integral for Simple Functions}
The Lebesgue integral of $f\in S^+$ with respect to the measure $\mu$:

\begin{equation}
\int_X f(x) d\mu(x) = \int_X f d\mu = I(f) = \sum_{i=1}^n c_i \mu(A_i)\ \in [0,\infty]
\end{equation}

This is a well-defined object that is independent of the specific representation of $f(x)$. It has the properties:

\begin{itemize}
\item $I(\alpha f + \beta g) = \alpha I(f) + \beta I(g)$ for $\alpha,\beta \geq 0$
\item $f \leq g \implies I(f) \leq I(g)$ (monotonicity)
\end{itemize}

The Lebesgue integral for simple functions enables defining the integral for more complex functions by approximating them.

\paragraph{Definition} 
Given a non-negative function $f:X \rightarrow [0,\infty)$, there are positive simple functions $h\in S^+$ that approximate it from below $\{h: h\in S^+, h\leq f\}$ with $h=\sum_{i=1}^n c_i \chi_{A_i}$. Then the integral of $f$ is given by the largest possible function within that set.

The \textit{Lebesgue Integral} of a function $f$ with respect to a measure $\mu$ is

\begin{equation}
\int_X f d\mu := \mathrm{sup}\{ I(h) : h\in S^+, h\leq f \}
\end{equation}

$f$ is called $\mu$-integrable if $\int_X f d\mu < \infty$.

The only thing that was needed to define the integral was a measure space $(X,\mathscr{A},\mu)$


\subsection{Monotone Convergence Theorem}

\subsubsection{Preliminaries}

Take the measurable positive function $f: X\rightarrow [0,\infty)$ which has lebesgue integral $\int_X f d\mu \in [0,\infty]$. 

\paragraph{Properties}

\subparagraph{Equality}
If $f=g$, $\mu$ almost everywhere, then $\int_X f d\mu = \int_X g d\mu$. 

That is, if $f=g$ "almost everywhere" with respect to the measure $\mu$, then the integrals are identical. This is more general than to simply say $f=g$. Rather, it requires $\mu(\{x \in X: f(x) \neq g(x) \}) = 0$. The Lebesgue integral "cannot see" things that happen on $0$-measure sets. For example, if the measure is a length measure $\mu ( [a,b] ) = b-a$, then, if $f$ and $g$ differ at a single point, then $\int_X f $ and $\int_X g$ are still the same. 

\subparagraph{Monotonicity}

Similartly, if $f\leq g$,  $\mu$ almost everywhere, then $\int_X f d\mu \leq \int_X g d\mu$.


Given that a positive simple function $h:X \rightarrow [0,infty)$ assumes a finite amount of different values, it is permissible to represent $h(x)$ in terms of the values $t$ that it assumes:

\begin{equation}
h(x) = \sum_{i=1}^n c_i \chi_{A_i} (x) = \sum_{t\in h(X)} t \chi_{\{x\in X: h(x) = t\}}
\end{equation} 

Using this representation, Then the Lebesgue integral:

\begin{equation}
I(h) = \sum_{t\in h(X)\setminus \{0\}} t \mu (\{ x\in X : h(x) = t \} ) 
\end{equation}

Where omitting the $t\in\{0\}$ makes no difference to the integral. Divide the set $X$ into $X = \tilde{X} + \tilde{X}^c$ with $\mu(\tilde{X}^c) = 0 $ and $\tilde{X} \in \mathscr{A}$. Then let:

\begin{equation}
\tilde{h}(X) := \left\{ \begin{array}{l} h(x), x\in \tilde{X}\\ a, x\in \tilde{X}^c \end{array} \right.
\end{equation}

Where $a \in [0, infty)$.  Then:

\begin{equation}
\tilde{h}(x) = \int_{t\in h(X)} t \chi_{x\in \tilde{X}: h(x) = t} + a \chi_{\tilde{X}^c}
\end{equation}

Then the integral:

\begin{equation}
I (\tilde{h}) = \sum_{t\in h(X)} t \mu( \{ x\in \tilde{X} : h(x) = t \}) + a \underbrace{\mu(\tilde{X}^c)}_{0}
\end{equation}

So that $I(h) = I(\tilde{h})$. This means that we can modify a simple function $h$ on a set with measure $0$ however we like without affecting the integral. This is enough to prove that if $f\leq g$ then $\int_X f d\mu \leq \int_X g d\mu$ in the "almost everywhere" sense. Divide $X$ into $\tilde{X}:= \{x \in X: f(x) \leq g(x) \}$ and $\tilde{X}^c$ with measure zero, where $f(x) \leq g(x)$ is not true. Since $I(h) = I(\tilde{h})$:

\begin{equation}
\begin{array}{rl}
\int_X f d\mu &= \mathrm{sup}\{ I(h): h\in S^+, h\leq f \} =  \mathrm{sup}\{ I(\tilde{h}): h\in S^+, \tilde{h}\leq f \leq g\mathrm{\ on\ }\tilde{X}\}\\
&\leq \mathrm{sup}\{ I(\tilde{h}): h\in S^+, \tilde{h}\leq g\mathrm{\ on\ }\tilde{X}\} = \int_X g d\mu
\end{array}
\end{equation}

\subparagraph{Zero Integral}
$f=0$, $\mu$ almost everywhere, then $\int_X d\mu = 0$. (Note that only positive functions $f$ are considered.




\subsubsection{Monotone Convergence Theorem}

Take a measure space $(X,\mathscr{A},\mu)$ and non-negative measurable functions $f_n: X \rightarrow [0,\infty), f: X \rightarrow [0,\infty)$, with $f_1 \leq f_2 \leq f_3 \leq ... $ with $\mu-a.e.$ (almost everywhere) and $\lim_{n\rightarrow \infty} f_n(x) = f(x). \mu-a.e. (x\in X)$.

The monotone convergence theorem states that, given a monotonic series of functions, the limit can be pulled into the integral:

\begin{equation}
\lim_{n\rightarrow \infty} \int_X f_n d\mu = \int_X f d\mu
\end{equation}


