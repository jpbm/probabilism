% \chapauthor{J. P. Balthasar Mueller}
\chapter{Linear Algebra}

\begin{multicols}{2}[\subsubsection*{Contents of this chapter}]
   \printcontents{}{1}{\setcounter{tocdepth}{2}}
\end{multicols}

\section{Spectral Theorem}

\section{Singular Value Decomposition}

\section{Generalized Eigenvectors}

\section{Multilinear Maps}

A multilinear map is a more general case of a linear map, which acts more than one vector in a way that is linear in each of its arguments.

\begin{equation}	
f: V_1 \times V_2 \times ... \times V_n \rightarrow W
\end{equation}

Where $V_1, V_2, ... , V_n$ and $W$ are vector spaces. An example would be the addition or subtraction of two or more vectors.

\section{Multilinear Forms}
Multilinear forms are multilinear maps that have a scalar output. An example is the dot product between two vectors, or summing over the elements of one or more vectors.

\begin{equation}
f: V_1 \times V_2 \times ... \times V_n \rightarrow K
\end{equation}

Where $V_1, V_2, ... , V_n$ and $K$ is a scalar field.


\section{Taking Derivatives}

\begin{equation}
\begin{array}{l}
\frac{d}{d\mathbf{x}} \left(u^Tx\right) = \left[\frac{d}{dx_1}\left(\sum_i u_i x_i\right),...,\frac{d}{dx_n}\left(\sum_i u_i x_i\right)\right] = u^T\\
\\
\frac{d}{d\mathbf{x}} \left(x^Tu\right) = \left[\frac{d}{dx_1}\left(\sum_i u_i x_i\right),...,\frac{d}{dx_n}\left(\sum_i u_i x_i\right)\right] = u^T\\
\\
\frac{d}{d\mathbf{x}} \left(x^Tx\right) = \left[\frac{d}{dx_1}\left(\sum_i x_i^2\right),...,\frac{d}{dx_n}\left(\sum_i x_i^2\right)\right] = 2x^T\\
\\
\frac{d}{d\mathbf{x}} \left(\mathbf{A}x\right) = \left[
\begin{array}{ccc} 
\underbrace{\frac{d}{dx_1}\left(\sum_i A_{1i} x_i\right)}_{A_{11}} &...& \underbrace{\frac{d}{dx_n}\left(\sum_i A_{1i} x_i\right)}_{A_1n}\\
\vdots&\vdots&\vdots\\
\underbrace{\frac{d}{dx_1}\left(\sum_i A_{ni} x_i\right)}_{A_{n1}} &...& \underbrace{\frac{d}{dx_n}\left(\sum_i A_{ni} x_i\right)}_{A_{nn}}\\
\end{array}\right] = \mathbf{A}\\
\end{array}
\end{equation}


\chapauthor{}
