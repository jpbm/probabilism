% \chapauthor{J. P. Balthasar Mueller}
\chapter{Linear Regression}
\label{chap:linearregression}

\begin{multicols}{2}[\subsubsection*{Contents of this chapter}]
   \printcontents{}{1}{\setcounter{tocdepth}{2}}
\end{multicols}

There is no escaping from linear regression but hardly anyone seems to agree on how to do it properly. Even worse, people seem to expect you to know things like the normal equations, which you'll never, ever need outside of a job interview. 


\section{Least Squares Regression ($L^2$)}

\subsection{The Normal Equations, Analytical Least Squares Estimator}
Section \ref{sec:linearequations} discussed the case of the linear system of equations:

\begin{equation}
\mathbf{A}\mathbf{x} = \mathbf{b}
\end{equation}


When $\mathbf{A}^{m\times n}$ with $m>n$, so that the system is overdetermined. This is, of course, the starting point for least squares regression, only that the convention is to use different letters:

\begin{equation}
\mathbf{X}\mathbf{\beta} = \mathbf{y}
\end{equation}

And that, seeing that the system is overdetermined, one looks for an approximate solution $\mathbf{\hat{\beta}}$, so that 

\begin{equation}
\mathbf{X}\mathbf{\beta} + \mathbf{\epsilon} = \mathbf{y}
\end{equation}


A natural approach for picking an approximate solution $\mathbf{\hat{\beta}}$ is to look for the projection of $\mathbf{y}$ in the column space of $\mathbf{X}$. That is, since the column rank $\leq n$ of $\mathbf{X}$ is insufficient to express $m$-dimensional $\mathbf{y}$ exactly in terms of only $m$ coefficients $\mathbf{\beta}$, we look for the $n$-dimensional shadow $\mathbf{\hat{\beta}}$ of some hypothetical higher dimensional exact solution. 

The projection has the property that it maximizes the dot product $(\mathbf{X}\mathbf{\hat{\beta}})\cdot \mathbf{y}$, and hence minimizes the length of the difference vector $\epsilon$. In turn, the length of the difference vector $\epsilon$ is $\sqrt{\epsilon\cdot\epsilon}$, which is monotonic to $\epsilon\cdot\epsilon = \sum^m_i \epsilon_i^2$. That means that finding the projection of $\mathbf{y}$ in the column space of $\mathbf{X}$ minimizes the  $L_2$ norm of $\epsilon$, also known as \textit{least squares error}.

There are two ways to go about finding $\mathbf{\hat{\beta}}$.

\subsection{The Quick Way to $\mathbf{\hat{\beta}}$}

By construction, the vector $\epsilon$ is orthogonal to the column space of $\mathbf{X}$. Which means:

\begin{equation}
\begin{array}{rl}
\mathbf{X}^T\epsilon &= 0\\
\mathbf{X}^T\left(\mathbf{X}\mathbf{\hat{\beta}}-\mathbf{y}\right) &= 0\\
\mathbf{X}^T\mathbf{X}\mathbf{\hat{\beta}} &= \mathbf{X}^T\mathbf{y}\\
\mathbf{\hat{\beta}} &= \left(\mathbf{X}^T\mathbf{X}\right)^{-1}\mathbf{X}^T\mathbf{y}
\end{array}
\end{equation}

Making use of the fact that $\mathbf{X}^T\mathbf{X}$ is square and therefore hopefully invertible.

\subsection{The Long Way to $\mathbf{\hat{\beta}}$}

Loss functions play a central role in computational statistics (for example when regularization is introduced), and therefore it is of interest to approach finding $\mathbf{\hat{\beta}}$ by instead minimizing the least square error. This requires:

\begin{equation}
\frac{d}{d\mathbf{\hat{\beta}}}L_2(\epsilon) = 0
\end{equation}

where

\begin{equation}
\begin{array}{rl}
L_2(\epsilon) &= \left(\mathbf{X}\mathbf{\hat{\beta}}-\mathbf{y}\right)^T\left(\mathbf{X}\mathbf{\hat{\beta}}-\mathbf{y}\right)\\
&= \mathbf{\hat{\beta}}^T\mathbf{X}^T\mathbf{X}\mathbf{\hat{\beta}} - \mathbf{\hat{\beta}}^T\mathbf{X}^T\mathbf{y} - \mathbf{y}^T\mathbf{X}\mathbf{\hat{\beta}} + \mathbf{y}^T\mathbf{y}
\end{array}
\end{equation}

Taking derivatives with respect to a vector is covered in section \ref{sec:derivatives}.

It follows:

\begin{equation}
\begin{array}{l}
\frac{d}{d\mathbf{\hat{\beta}}}\left(x^T\mathbf{X}^T\underbrace{\mathbf{X}\mathbf{\hat{\beta}}}_{u(\mathbf{\hat{\beta}})}\right) = \frac{d}{du}\left(u^Tu\right)\frac{d}{d\mathbf{\hat{\beta}}}u = 2u^T\frac{d}{d\mathbf{\hat{\beta}}}u = 2\mathbf{\hat{\beta}}^T\mathbf{X}^T\mathbf{X}\\
\\
\frac{d}{d\mathbf{\hat{\beta}}}\mathbf{\hat{\beta}}^T\mathbf{X}^T\mathbf{y} = \mathbf{y}^T\mathbf{X}\\
\\
\frac{d}{d\mathbf{\hat{\beta}}}\mathbf{y}^T\mathbf{X}\mathbf{\hat{\beta}} = \mathbf{y}^T\mathbf{X}\\
\\
\frac{d}{d\mathbf{\hat{\beta}}}\mathbf{y}^T\mathbf{y} = 0
\end{array}
\end{equation}

So that	

\begin{equation}
\begin{array}{rl}
\frac{d}{dx}L_2(\epsilon) = 0 &= 2\mathbf{\hat{\beta}}^T\mathbf{X}^T\mathbf{X} - 2\mathbf{y}^T\mathbf{X}\\
\mathbf{\hat{\beta}}^T\mathbf{X}^T\mathbf{X} &= \mathbf{y}^T\mathbf{X}\\
\mathbf{X}^T\mathbf{X}\mathbf{\hat{\beta}} &= \mathbf{X}^T\mathbf{y}\\
\mathbf{\hat{\beta}} &= \left(\mathbf{X}^T\mathbf{X}\right)^{-1}\mathbf{X}^T\mathbf{y}
\end{array}
\end{equation}



\subsection{Projection Matrix}

If $\mathbf{\hat{y}}=\mathbf{X}\mathbf{\hat{\beta}}$ is the projection of $\mathbf{y}$ in the column space of $\mathbf{X}$, then, based on the result for $\hat{\beta}$, the projection matrix is $\mathbf{P} = \mathbf{X}\left(\mathbf{X}^T\mathbf{X}\right)^{-1}\mathbf{X}^T$. In a fully determined system, $\mathbf{P}=\mathbf{I}$. Projection matrices have eigenvalues that are either $1$ or $0$, corresponding to dimensions that are kept or discarded during the projection operation.


\subsection{Bayesian Perspective on Least Squares Regression}




\subsection{Q-plots}
\subsection{Variance Inflation Factor}

\section{Total Least Squares}
While least squares regression only allows for errors in the dependent variable, total least squares regression allows for measurement errors on both variables.

\section{Ridge Regression (Tikhonov Regularization, $\lambda ||\mathbf{\beta}||^2$)}
Ridge Regression ads the $L^1$ norm of the weight vector to 

\subsection{Analytical Ridge Estimator}
\subsection{Bayesian Perspective on Ridge Regression}

\section{Least Absolute Shrinkage and Selection Operator Regression (LASSO)}


\section{Least Absolute Deviation Regression (LAD, $L^1$)}


\section{Generalized Linear Models}

\section{Count Regressions}

Count data is common, and often incorrectly analyzed using ordinary least squares. In general, least squares is only applicable when the error on the data, or a transformation of the data, has normal distribution with uniform variance. Count data tends to defy that expectation, especially for low counts. Count data also tends to be heteroskedastic: higher counts tend to fluctuate more widely.

\begin{itemize}
\item The data is strictly positive. OLS models can predict negative values.
\item Especially for low counts, the distribution is highly skewed
\item The data is discrete, not continuous.
\item The fluctuation on the mean is not constant. The data is heteroskedastic. 
\end{itemize}

The definite resource for modeling count data is \citeasnoun{hilbe2014modeling}.


\subsection{Poisson Regression Models}
Poisson models tend to be the first pass at modeling count data, perhaps analogous to how ordinary least squares (OLS) regression is the first pass at modeling continuous data. Poisson models distinguish themselves, in general, by assuming that the fluctuation of the data about the mean has Poisson distribution.


\subsubsection{``Ordinary" Poisson Regression}
The term Poisson regression tends to be used interchangeably with a specific type of model, which is a generalized linear model with Poisson noise and log link. A concise reference for those models is chapter 4 in \citeasnoun{rodriguez2007generalized}.

Specifically, the likelihood of observing a value $y$ is assumed to follow a Poisson distribution:

\begin{equation}
p(y|\mu) = \frac{e^{-\mu} \mu^{y}}{y!}
\end{equation}

Where the only parameter, $\mu$, is both the mean and the variance of the Poisson distribution. The model acquires additional structure when $\mu$ is assumed to be a function of some explanatory variables, $x$, i.e. $\mu = f(x)$. The canonical Poisson regression uses a log-linear relationship between the coefficients and the mean, i.e. $\mu = e^{\mathbf{\beta \cdot x}}$. The result is a generalized linear model with Poisson error and link log.

The assumptions of this "ordinary Poisson regression" (my idea to call it that) are: 
\begin{itemize}
\item The error has Poisson distribution.
\item The data is strictly positive.
\item The data has discrete distribution (though the generalization to continuous numbers is pretty trivial)
\item The data is i.i.d., meaning that the observed count events are results of independent trials. (Example: the number of kids in a family is unaffected by the number of kids the neighbors have)
\end{itemize}

And, very importantly:
\begin{itemize}
\item It is a log-linear model! The relationship between the dependent variable ($Y$) and the independent variables is log linear. I.e. $ln(Y)$ is a linear function of the coefficients.
\item The distribution of the data is heteroskedastic so that the mean equals the variance. The Poisson distribution only has one parameter and (i.e. $\mu = \sigma$)!
\end{itemize}

In the model, the likelihood of observing a value $y$ is assumed to follow a Poisson distribution:

\begin{equation}
p(y|\mu) = \frac{e^{-\mu} \mu^{y}}{y!}
\end{equation}

Where the only parameter, $\mu$, is both the mean and the variance of the Poisson distribution. The model acquires additional structure when $\mu$ is assumed to be a function of some explanatory variables, $x$, i.e. $\mu = f(x)$. The canonical Poisson regression uses a log-linear relationship between the coefficients and the mean, i.e. $\mu = e^{\mathbf{\beta \cdot x}}$. The result is a generalized linear model with Poisson error and link log.

\begin{equation}
p(y|\mathbf{x};\mathbf{\beta}) = \frac{e^{-\exp(\mathbf{\beta\cdot x})} (\exp(\mathbf{\beta\cdot x}))^{y}}{y!}
\end{equation}


\begin{figure}
\centering
    \includegraphics[width=\textwidth]{poissonmodels02.png}
    \caption{Left: Ordinary Poisson Regression on data that satisfies the assumptions of the model. For small rates, the model is approximately linear, but beyond that it shows exponential growth because of the log-link. Right: Residuals showing heteroskedasticity with variance increasing with the mean. Dividing the residuals by the mean yields constant variance.}
    \label{fig:ordinarypoisson}
\end{figure}


\subsubsection{Poisson Noise \& Central Limit Theorem}

For small count rates, the Poisson distribution is highly skewed and strictly positive. For large count rates, the Poisson distribution is essentially normal, except that variance and mean are locked.

If you are dealing with large counts, then the Poisson model still has the feature of being heteroskedastic.

\subparagraph{Advantages of using Poisson Noise}

\begin{itemize}
\item The Poisson Distribution is highly skewed for small rates, and strictly positive. For high enough count rates, this advantage disappears.
\item The Poisson Distribution is heteroskedastic
\end{itemize}

\subparagraph{Drawbacks of using Poisson Noise}

\begin{itemize}
\item The Poisson distribution only has a single parameter. The assumption that the mean and the variance are the same is very restrictive.
\end{itemize}


\begin{figure}
\centering
    \includegraphics[width=\textwidth]{poissonmodels01.png}
    \caption{Left: The Poisson Distribution for small rates is highly skewed. Right: The Poisson Distribution for intermediate rates looks very similar to a normal distribution. In both cases, the variance and the mean are the same value.}
    \label{fig:poissonandcentrallimit}
\end{figure}


\subsubsection{Poisson Noise Analogy to Least Squares Regression}
To anchor intuition in familiar territory, consider least squares regression with a log-linear relationship between endogenous and exogenous variables (that is, the model assumes a relationship of $log(y)=a + \mathbf{bx}$ that is beset with Gaussian noise). The familiar form for the model is:

\begin{equation}
\begin{array}{rl}
y &= a\exp{\mathbf{b \cdot x}} + \epsilon \\
&= \exp\mathbf{\beta \cdot x} + \epsilon
\end{array}
\end{equation}

Where the constant $a$ was absorbed into the coefficient vector $\mathbf{\beta}$ in the second line, and $\mathbf{x} \rightarrow [1,\mathbf{x}]$. $\epsilon$ is an error term that is assumed to have normal distribution with zero mean, i.e. $\epsilon \sim \mathscr{N}(0,\sigma)$. It's a bit unnatural, but this can be rewritten, absorbing the parameters into the random term:

\begin{equation}
y = 0 + \epsilon'
\end{equation}

With $\epsilon' \sim \mathscr{N}(\mu = \exp{\mathbf{\beta \cdot x}},\sigma)$. Now what if the fluctuations aren't normally distributed about the mean, but they are Poisson distributed about the mean? In that case, $\epsilon' \sim \mathrm{Poisson}(\mu = \exp{\mathbf{\beta \cdot x}})$. 



\subsubsection{Multivariate Poisson Model}
For least squared regression, it's totally common to look at multivariate models with interesting codependence structure captured by a covariance matrix. In analogy to that, there is multivariate Poisson Regression. The math looks quite different. 

\url{http://www2.stat-athens.aueb.gr/~karlis/multivariate%20Poisson%20models.pdf}

\subsubsection{Goodness of Fit}
The Poisson deviance is given by:

\begin{equation}
D = 2\sum\left\{ y_i \log\left( \frac{y_i}{\hat{\mu}_i} \right) - \left( y_i - \hat{\mu}_i \right) \right\}
\end{equation}

Here, $\hat{\mu}_i = e^{\mathbf{x}^T_i \hat{\mathbf{\beta}}}$ is the fitted mean of the $i$th data point, and $y_i$ is the observed count of the $i$th datapoint. 

For large sample sizes, the deviance will be distributed approximately chi-squared with $n-p$ degrees of freedom, where $n$ is the number of data points and $p$ is the number of features. An alternative to the deviance is Pearson's chi-squared statistic. 

\subsubsection{General $\mu = f(\mathbf{x})$}
The Poisson-ness only has to do with how the data fluctuates about the mean. How the mean is expected to depend on the explanatory variables is another question. In so far, general other functions, including highly-nonlinear machine learning models, can be fit under the assumption of Poisson noise.

XGBoost supports poisson loss. Empirically, it seems that Poisson loss performs worse in the regime where the model is underfitting, and slightly better in the regime where the model is overfitting. The difference is especially pronounced in the Poisson deviance and for sparse data.


\begin{figure}
\centering
    \includegraphics[width=\textwidth]{poissonmodels03.png}
    \caption{Left: Ordinary Poisson Regression on data that satisfies the assumptions of the model. For small rates, the model is approximately linear, but beyond that it shows exponential growth because of the log-link. Right: Residuals showing heteroskedasticity with variance increasing with the mean. Dividing the residuals by the mean yields constant variance.}
    \label{fig:ordinarypoisson}
\end{figure}

\section{Logistic Regression}

Logistic Regression is a generalized linear model suitable for fitting probabilities, i.e. dependent variables that vary $y\in (0,1)$. 

\subsection{Binary Response Case}
Assume that the dependent variable is a scalar and not a vector (i.e. only a single probability is being predicted). Then, assume that the dependence of that probability on the exogenous variable $\mathbf{x}$ and the parameters of the model $\mathbf{b}$ is given by:

\begin{equation}
p(\mathbf{x}) = \frac{1}{1+\exp(\mathbf{b\cdot x})}
\end{equation}

This is, of course, the sigmoid function, which also shows up as the Fermi-Dirac distribution. It creates a symmetric mapping from the interval $(-\infty,\infty)$ to $(0,1)$. It amounts to treating the log-odds as a linear function of the coefficients: 

\begin{equation}
\ln\left(\frac{p(\mathbf{x})}{1-p(\mathbf{x})}\right) = \mathbf{b \cdot x}
\end{equation}

Equivalently:

\begin{equation}
\frac{p(\mathbf{x})}{1-p(\mathbf{x})} = \prod_j e^{b_j x_j}
\end{equation}

So the model assumes that different values of the independent variables will have a multiplicative effect on the likelihood ratio of the two binary outcomes. There is a deep reason for this choice of mapping, which is why it shows up as the Fermi-Dirac distribution in physics. In physics, it emerges from looking at the probability of energy levels being occupied by fermions. This is done by considering the number of equally possible configurations that correspond to a set of probabilities of a certain energy level being occupied. The Fermi-Dirac distribution is the distribution that maximizes the number of equally possible configurations, and so I suppose the sigmoid function has some maximum entropy property, which makes it the "maximally naive guess". 

Assuming that the data is generated by independent trials, the likelihood of observing the data is given by multiplying the probbility of all the datapoints together:

\begin{equation}
\mathscr{L}(\mathrm{data}|p(\mathbf{x})) = \prod_{y_i=1} p(\mathbf{x_i}) \prod_{y_i=0} (1-p(\mathbf{x_i}))
\end{equation}

Strictly speaking, I think there should be a factor of $1/N!$ here to account for the probability of seeing a specific permutation of the individual data points (which are distinguishable), but that factor will be independent of $p(\mathbf{x})$ and so it will not matter to estimation. In the literature I haven't seen a factor show up.

The expression $-2 \ln(\mathscr{L})$ is called the \textit{deviance}, which is the analog to residual sum of squares in ordinary linear regression. In OLS, the log likelihood is: $\ln\mathscr{L} = -\sum|\hat{y}-y|^2 + ...$, while the deviance is $-2\ln\mathscr{L} = -\sum_i \ln\left[\left(p(\mathbf{x_i})^{y_i}(1-p(\mathbf{x_i}))^{y_i-1}\right)^2\right]$.

Logistic regression is then fit by finding the coefficients that minimize the deviance / maximize the likelihood.

Question for another day: in OLS, there is an assumption that the data will deviate from the mean, where the probability of observing a certain deviation is given by a normal distribution. This has the consequence, for example, that outliers have an outsize impact on the fitted mean, because the normal distribution otherwise models them as exceedingly unlikely. If the normal distribution is the error distribution for OLS, what is the error distribution for logistic regression? I believe it looks like the Fermi-Dirac Distribution, where the Fermi-Energy is the 50\% decision boundary. 

\subsection{Pseudo-$R^2$}
The deviance can be used to create an error metric analogous to $R^2$ for OLS, named Pseudo-$R^2$. It compares the deviance to the deviance of simply assuming the average probability based on the class balance.

\begin{equation}
\mathrm{pseudo}-R^2 = 1 - \frac{\mathrm{deviance}}{\mathrm{deviance\ when\ }p=\mathrm{\ average\ class}}
\end{equation}





\chapauthor{}

