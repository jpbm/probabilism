\section{Types of Transformations}

% Similarity
\subsection{Similarity Transformations}
\label{sec:similaritytrans}

If $\mathbf{T}$ is a nonsingular matrix, then a similarity transformation is defined as:

\begin{equation}
\mathbf{A} = \mathbf{TBT^{-1}}
\end{equation}

And $\mathbf{A}$ and $\mathbf{B}$ are said to be \textit{similar}.

% Affine 
\subsection{Affine Transformations}
\label{sec:affine}

Affine transformations are the combination of a linear map and a translation, which has the form $f(\mathbf{x}) = \mathbf{A}\mathbf{x} + \mathbf{b}$. 

\begin{equation}
f: V \rightarrow W
\end{equation}

Where $V$ and $W$ are vector spaces. Affine transformations can be expresses as matrices by adding an entry with a constant to the vectors that describe a point in space. For example, for $\mathbf{x} \in \mathbb{R}^n$,  the affine transform $f(\mathbf{x}) = \mathbf{A}\mathbf{x} + \mathbf{b}$ with $A\in\mathbb{R}^{n,n}$ and $x,b \in \mathbb{R}^{n}$ can be expressed as the product of a rectangular matrix $\mathbf{M}$ and a vector $\mathbf{c}$ as:

\begin{equation}
\mathbf{A}\mathbf{x} + \mathbf{b} = \underbrace{\left[\begin{array}{c|c} \mathbf{A} & \mathbf{b} \end{array}\right]}_{\mathbf{M}} \underbrace{\left[\begin{array}{c} \mathbf{x} \\ 1\end{array} \right]}_{\mathbf{c}}
\end{equation}

Where $\mathbf{c}^T = \left[x_1,x_2,x_3,...,x_n,1\right]$ and $\mathbf{M} \in \mathbb{R}^{n,n+1}$.


\subsection{Unitary Transformations}
Unitary transformations are transformations that preserve the inner product, i.e. $\hat{U}x \cdot \hat{U}y = x \cdot y$. As linear transformations, they are represented by unitary matrices (cf. section \ref{sec:unitary}). Unitary transformations include translations, reflections and rotations. 


\subsection{Multilinear Maps}

A multilinear map acts on several vectors in a way that is linear in each of its arguments. A $k$-linear map acts on $k$ vectors, where $k=2$ are bilinear maps and $k=1$ are linear maps.

\begin{equation}	
f: V_1 \times V_2 \times ... \times V_n \rightarrow W
\end{equation}

Where $V_1, V_2, ... , V_n$ and $W$ are vector spaces. An example would be the addition or subtraction of two or more vectors.

\subsection{Multilinear Forms}
Multilinear forms are multilinear maps that have a scalar output. An example is the dot product between two vectors, or summing over the elements of one or more vectors.

\begin{equation}
f: V_1 \times V_2 \times ... \times V_n \rightarrow K
\end{equation}

Where $V_1, V_2, ... , V_n$ and $K$ is a scalar field.

