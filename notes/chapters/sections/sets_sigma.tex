\section{$\sigma$-Algebras}

\subsection{Definition}
Given a set $X$, a $\sigma$-Algebra $\mathscr{A}$ is a collection of subsets of a given set $X$, which has to satisfy the conditions:


\begin{itemize}
\item $\emptyset, X \in \mathscr{A}$
\item If $A\in\mathscr{A}$, then $A^c := X\setminus A \in \mathscr{A}$
\item $\mathscr{A}$ has (possibly infinitely many) countable subsets $A_i \in \mathscr{A},\ i \in \mathbb{N}$. Then $\bigcup_{i=1}^{\infty}A_i \in \mathscr{A}$. That is, the $\sigma$-algebra is closed under countable unions of its subsets. 
\end{itemize}

An element of the $\sigma$-Algebra are $\mathscr{A}$-measurable sets, or measurable with respect to the $\sigma$-Algebra $\mathscr{A}$. A $\sigma$-algebra is a subset of the power set of $S$, i.e. $\mathscr{A} \subseteq P(X)$. 

\subsubsection{Example: The Smallest Possible $\sigma$-Algebra}

\begin{equation}
\mathscr{A} = \{\emptyset, X\}
\end{equation}

\subsubsection{Example: The Largest Possible $\sigma$-Algebra} 
The largest possible $\sigma$-Algebra is the power set:
\begin{equation}
\mathscr{A} = P(X)
\end{equation}

However, there are important examples where it is not possible to define a $\sigma$-algebra on the full power set. 


\subsection{Intersection Property}
The intersection of $\sigma$ algebras is also a $\sigma$-algebra. That way, a $\sigma$-algebra with the desired properties can be constructed by creating individual $\sigma$-algebras with the properties in question and forming their intersection.

\begin{equation}
\bigcap_i \mathscr{A} \ \ \mathrm{is\ also\ a\ }\sigma-\mathrm{algebra}
\end{equation}


\subsection{Generated $\sigma$-Algebras}

For a subset of the powerset $\mathscr{M} \subseteq P(X)$, that does not necessarily have to satisfy the properties of the $\sigma$-Algebra, the \textit{smallest} $\sigma$-algebra that contains $\mathscr{M}$ is the $\sigma$-algebra \textit{generated} by $\mathscr{M}$. It can be constructed through the intersection of all $\sigma$-algebras  on $X$ that contain $\mathscr{M}$. 

\begin{equation}
\sigma(\mathscr{M}) = \bigcap_{\mathscr{A}\supseteq\mathscr{M}}\mathscr{A}
\end{equation}

$\sigma(\mathscr{M})$ is the $\sigma$-algebra \textit{generated} by $\mathscr{M}$. 

\subsubsection{Example: Generated $\sigma$-Algebra}

Take $X = \{a,b,c,d\}$, and $\mathscr{M} = \{ \{a\}, \{b\}\}$. Note that $\mathscr{M}$ is not a $\sigma$-algebra. To find the smallest $\sigma$-algebra that contains $\mathscr{M}$, one adds the elements necessary to fulfill the conditions on a $\sigma$-algebra. These are the empty set $\emptyset$ and the full set $X$, the union $\{a,b\}$, and the complements.

\begin{equation}
\sigma(\mathscr{M}) = \left\{ \emptyset, X, \{a\}, \{b\}, \{a,b\}, \{b,c,d\}, \{a,c,d\}, \{c,d\} \right\}
\end{equation}


\subsection{Borel $\sigma$-Algebras}

Borel $\sigma$-Algebras is the $\sigma$-Algebra generated by \textit{open sets}, for example $\sigma(\mathbb{R}^n)$ or $\sigma(\mathscr{M})$ with $\mathscr{M} = (0,1)$.

\subsection{Measures, Measure Spaces}
A \textit{measure} is a function that gives a volume measure for subsets of a $\sigma$-algebra, where a $\sigma$-algebra is a special sort of collection of subsets of some set $X$. 
The combination $(X,\mathscr{A})$ of a set $X$ and a $\sigma$-Algebra on $X$ is called a \textit{measurable space}. A measure is a function defined on the $\sigma$-Algebra and maps to the positive real line (including $\infty$!):

\begin{equation}
\mu: \mathscr{A} \rightarrow [0,\infty)\cup\{\infty\} 
\end{equation}

It has to satisfy the conditions:

\begin{itemize}
\item $\mu(\emptyset) = 0$
\item $\mu(\bigcup_i A_i) = \sum_i \mu(A_i)$ if $A_i \cap A_j = \emptyset$ when $i\neq j$ (additive)
\item $\mu(\bigcup^{\infty}_i A_i) = \sum^{\infty}_i \mu(A_i)$ if $A_i \cap A_j = \emptyset$ when $i\neq j$ ($\sigma$-additive) 
\end{itemize}

Where the infinite sum corresponds to gradually approximating the full volume by taking the union of countably infinitely many subsets. The collection $(X,\mathscr{A},\mu)$ is a \textit{measure space}.

\subsubsection{Example: Counting Measure}

\begin{equation}
\mu(A) = \left\{\begin{array}{l} |A| \mathrm{\ if\ }A\mathrm{\ has\ finite\ elements}\\ \infty\mathrm{\ else}\end{array}\right.
\end{equation}

\subsubsection{Example: Dirac Measure}
\begin{equation}
\delta_p(A) = \left\{\begin{array}{l} 1 \mathrm{\ if\ }p\in A\\ 0 \mathrm{\ else} \end{array}\right.
\end{equation}

\subsection{Normal Volume Measure}
For $X=\mathbb{R}^n$, the conventional volume measure satisfies the properties of:

\begin{itemize}
\item $\mu([0,1]^n) = 1$
\item $\mu(x + A) = \mu(A)$ for some $x\in \mathbb{R}^n$ (translation invariance)
\end{itemize}
