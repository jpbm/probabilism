\section{Sets}

Sets are a term to describe collections of things. The things could be countable objects, such as the integers between $1$ and $10$, or contain a continuum, for example all the real numbers between $1$ and $10$.

\subsection{Set Operations}

\subsection{Common Sets}

\subsection{DeMorgan's Rules}
\label{sec:demorgan}
DeMorgan's Rules relate the complement of the union to the intersection of the complements, and the complement of the intersection to the union of the complements.

\label{sec:demorgan}

\begin{equation}
\left(\bigcup_{i\in\{i\}}A_i\right)^c = \bigcap_{i\in\{i\}}A^c_i
\end{equation}

\begin{equation}
\left(\bigcap_{i\in\{i\}}A_i\right)^c = \bigcup_{i\in\{i\}}A^c_i
\end{equation}

\subsection{Inclusion - Exclusion Principle}
\label{inclusionexclusion}

The inclusion-exclusion principle is used to calculate the size of the union of sets. This requires counting each region of some complicated overlapping Venn diagram exactly once, which, in turn requires accounting for overcounting wherever sets overlap. Let $\{A_i | i\in \{i\}_n \}$ be a collection of $n$ overlapping sets indexed by $i\in \{i\}_n$, then the inclusion-exclusion principle is given by:

\begin{equation}
\left|\bigcup_{i\in\{i\}_n} A_i\right| = \sum^n_{k=1} (-1)^{k-1} \sum_{\{j\}_k \subseteq \{i\}_n} \left|\bigcap_{j\in\{j\}_k} A_j\right|
\end{equation}


Where the sum over $\{j\}_k \subseteq \{i\}_n$ is over all $k$-element subsets of $\{i\}_n$.

\subsubsection{Example: n=2 Sets and n=3 Sets}

\subparagraph{n=2}
\begin{equation}
\begin{array}{rl}
\left|\bigcup_{i\in\{1,2\}}A_i\right| =&  \sum^2_{k=1} (-1)^{k-1} \sum_{\{j\}_k \subseteq \{1,2\}} \left|\bigcap_{j\in\{j\}_k} A_j\right|\\
=&(-1)^{0}\left(|A_1| + |A_2|\right) \\
&+ (-1)^{1}\left(|A_1 \cap A_2| \right)
\end{array}
\end{equation}


\subparagraph{n=3}
\begin{equation}
\begin{array}{rl}
\left|\bigcup_{i\in\{1,2,3\}}A_i\right| =&  \sum^3_{k=1} (-1)^{k-1} \sum_{\{j\}_k \subseteq \{1,2,3\}} \left|\bigcap_{j\in\{j\}_k} A_j\right|\\
=&(-1)^{0}\left(|A_1| + |A_2| + |A_3|\right) \\
&+ (-1)^{1}\left(|A_1 \cap A_2|  + |A_1 \cap A_3| + |A_2 \cap A_3|  \right) \\ 
&+ (-1)^{2}\left(|A_1 \cap A_2 \cap A_3|   \right)
\end{array}
\end{equation}

\subsubsection{Example: Counting Integers}

How many integers are there between 1 and 100 that are neither divisible by 3,5 nor 7?

Let $S$ be the set of all integers between 1 and 100. The size of the set is $|S| = 100$. The subset of $S$ that is numbers divisible by 3 is $A_3 \subseteq S$ with $|A_3| = 33$ because $100/3 = 33.\overline{333}$. Similarly, $|A_5| = 20$ and $|A_7| = 14$.  The set of integers that is not divisible by 3, 5 or 7 is:

\begin{equation}
S \setminus \bigcup_{i\in\{3,5,7\}} A_i
\end{equation}

So that the sought after quantity is :

\begin{equation}
\begin{array}{rl}
\left|S \setminus \bigcup_{i\in\{3,5,7\}} A_i\right| =& |S| - \left[ |A_3| + |A_5| + |A_7| \right.\\
& \left. - |A_3 \cap A_5| - |A_3 \cap A_7| - |A_5\cap A_7| + |A_3\cap A_5\cap A_7|\right]
\end{array}
\end{equation}

The size of the intersection $|A_3\cap A_5| = 6$ because 100 is 6 times divisible by $3\times 5 = 15$. Similarly, $|A_3 \cap A_7| =  4$, $|A_5 \cap A_7| =  2$ and $|A_3\cap A_5 \cap A_7| =  0|$. Hence:

\begin{equation}
\left|S \setminus \bigcup_{i\in\{3,5,7\}} A_i\right| = 100 - 33 - 20 - 14 + 6 + 4 + 2 - 0 = 45
\end{equation}

There are 45 integers between 1 and 100 that are not divisible by 3, 5 or 7.