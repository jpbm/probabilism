\section{Asymptotic Theory}

Asymptotic theory, large sample theory, or limit theory, deals with the question of the limiting behavior of sequences of random variables. 

The notion of convergence for random variables is more involved than is usual for, for example, a sequence of numbers or functions in calculus. There are different types of convergence, and they do not necessarilty imply one another.

The chain of implication is:

\begin{equation}
\mathrm{quadratic\ mean} \rightarrow \mathrm{probability} \rightarrow \mathrm{distribution}
\end{equation}

For point mass distributions only, convergence in distribution implies convergence in probability. Similarlty, implication rules exist for functions of random variables when convergence for the underlying random variables is given. See Theorems 5.4, 5.5 and 5.17 in \citeasnoun{wasserman2013all} (pages 73-74, 81). 

\subsection{Preasymptotics}
The behavior in the intermediate regime, for example where $n$ large but not yet asymptotic. Asymptotic behavior kicks in at very different rates for different underlying processes. The ``baseline" in terms of speed is represented by normally distributed random variables.


%convergence in probability
\subsection{Convergence in Probability}

\begin{equation}
\mathbb{P}(|X_n - X| > \epsilon) \rightarrow 0
\end{equation}

as $n\rightarrow \infty$. Convergence in Probability means that the distribution of $X_n$ becomes sharper and sharper around $X$ as $n\rightarrow \infty$. At $n=\infty$, it has point mass distribution concentrated at $X$.


% convergence in distribution
\subsection{Convergence in Distribution}
Where $F$ is the CDF of $X_n$,

\begin{equation}
\lim_{n\rightarrow \infty} F_n(t) = F(t)
\end{equation}

For all $t$ for which $F$ is continuous. That means, convergence is satisfied even when the equality is violated at points of discontinuity.


% convergence in quadratic mean
\subsection{Convergence in Quadratic Mean, Convergence in $L_2$}

\begin{equation}
\mathbb{E}(X_n - X)^2 \rightarrow 0
\end{equation}

as $n\rightarrow \infty$.


% almost sure convergence 
\subsection{Almost Sure Convergence}

$X_n$ converges \textit{almost surely} to $X$ if:

\begin{equation}
\mathbb{P}(\{\omega: X_n(\omega) \rightarrow X(\omega) \}) = 1
\end{equation}

Which I would read as the value of the measurable map $X_n$ converging to the value of the measurable map $X$ everywhere on the probability space except possibly on a set of probability measure $0$.


% l1 convergence
\subsection{$L_1$ Convergence}

$L_1$ convergence requires $\mathbb{E}|X_n - X| \rightarrow 0$ as $n\rightarrow 0$.



% WLLN
\subsection{Weak Law of Large Numbers}

The sample mean of i.i.d. variables  $\overline{X}_n$ converges in probability to the mean of the distribution $\mu$i. It can be proven with Chebyshev's inequality that:

\begin{equation}
\mathbb{P}(|\overline{X}_n-\mu|>\epsilon) \leq \frac{\mathbb{V}(\overline{X}_n)}{\epsilon^2}=\frac{\sigma^2}{n \epsilon^2}
\end{equation}

In words, the probability that the sample mean deviates from the population mean by more than $\epsilon$ has an upper bound that decreases inversely proportional to $n\epsilon^2$. The probability becomes more and more centered around the mean $\mu$.

% SLLN
\subsection{Strong Law of Large Numbers}
The strong law of large number gives almost surely convergence of the sample mean to the population mean.

If $\mathbb{E}|X_1| < \infty$, then $\overline{X_n}\xrightarrow{as}\mu$.



% CLT
\subsection{Central Limit Theorem}
The distribution of the sample mean converges in distribution to a normal distribution with variance $\sigma^2/n$ and mean $\mu$.

If $Z_n = \frac{\overline{X}_n - \mu}{\sqrt{\mathbb{V}(\overline{X}_n)}}$ then $\lim_{n\rightarrow \infty} \mathbb{P}(Z_n \leq z) = \Phi(z)$, where $\Phi(z)$ is the CDF of a standard normal distribution. 

It turns out that when $Z_n$ is obtained by normalizing not by the (most likely unknown) population variance $\sigma$ but by the sample variance $S_n^2$, the CLT still holds. The accuracy of this is given by the Berry-Ess\'een Inequality.


% multivariate CLT
\subsection{Multivariate Central Limit Theorem}
Given $\mathbf{X_1, ... ,X_n}$ i.i.d random vectors where each vector:

\begin{equation}
\mathbf{X_i} = \left(\begin{array}{c}X_{1i}\\ X_{2i} \\ \vdots \\ X_{ki} \end{array}\right)
\end{equation}

Then the population mean:

\begin{equation}
\mathbf{\mu} = \left(\begin{array}{c} \mu_1 \\ \mu_2 \\ \vdots \\ \mu_k \end{array} \right) =  \left(\begin{array}{c} \mathbb{E}(X_{i1}) \\ \mathbb{E}(X_{i2}) \\ \vdots \\ \mathbb{E}(X_{ik}) \end{array} \right)
\end{equation}

The variance is given by the matrix $\Sigma$ as before. The sample mean:

\begin{equation}
\overline{\mathbf{X}} = \left(\begin{array}{c}\overline{X_{1}}\\ \overline{X_{2}} \\ \vdots \\ \overline{X_{k}} \end{array}\right)
\end{equation}

Then $\sigma^{-\frac{1}{2}} (\overline{X}-\mu)$ converges in distribution ot $\mathscr{N}(0,1)$.


% proof
\subsection{Proof of the Central Limit Theorem}
\citeasnoun{wasserman2013all}, page 81.

Given i.i.d random variables $X_i$, the transformation $Y_i = \frac{X_i-\mu}{\sigma}$ gives i.i.d. random variables with zero mean and unit variance. Let $\psi(t)$ be the MGF of $Y_i$. Since $Y_i$ are i.i.d., the sum $\sum_{i=1}^n Y_i$ has MGF $\psi(t)^n$. The normalized sample mean $Z_n = \frac{1}{\sqrt{n}}\sum_{i=1}^n Y_i$ has MGF $\Xi_n(t)=\psi(t/\sqrt{n})^n$. Two random variables that have the same MGF in an open interval about the point $t=0$ have the same distribution, probably because the Laplace transform is injective. Therefore, if $\psi_n(t)\rightarrow \psi(t)$ in some open interval around $t=0$, then their underlying random variables $Z_n \xrightarrow{dist}Z_n$ converge in distribution. Taking the Taylor expansion of $\epsilon_n(t)$:

\begin{equation}
\epsilon_n(t) = \left(1+0+\frac{t^2}{2! n} + ... \right)^n \rightarrow e^{t^2/2}
\end{equation}

Which is the MGF of $\mathscr{N}(0,1)$

% Delta Method
\subsection{Delta Method}

The delta method allows statements regarding the convergence of functions of random variables, whenever the input random variable converges in distribution to a normal distribution. 

If $Y_n$ has a limiting normal distribution, and $g(Y_n )$ is a smooth function so that $g'(\mu) \neq 0$, then if:

\begin{equation}
\frac{\sqrt{n}(Y_n - \mu)}{\sigma} \xrightarrow{dist} \mathscr{N}(0,1)
\end{equation}

Then:

\begin{equation}
\frac{\sqrt{n}(g(Y_n)-g(\mu))}{|g'(\mu)|\sigma}\xrightarrow{dist}\mathscr{N}(0,1)
\end{equation}

Rewriting, if $Y_n \xrightarrow{dist}\mathscr{N}(\mu,\frac{\sigma^2}{n})$ then $g(Y_n)\xrightarrow{dist}\mathscr{N}(g(\mu),(g'(\mu))^2\frac{\sigma^2}{n})$.

\subsubsection{Multivariate Delta Method}

If $\mathbf{Y_n} \xrightarrow{dist}\mathscr{\mathbf{\mu},\mathbf{\Sigma}}$ then the scalar-valued function $g(\mathbf{Y_n}) \xrightarrow{dist}\mathscr{N}(g(\mathbf{\mu}),\frac{1}{n} (\nabla g(\mu))^T \mathbf{\Sigma} (\nabla g(\mu)) )$.

Use case would be functions of several sample means, where the underlying samples have non-trivial covariance (cf. \cite{wasserman2013all}, p. 80).

