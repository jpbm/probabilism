\section{$\mathbf{A} = \mathbf{U\Sigma V}^{\dagger}$ Singular Value Decomposition}
\label{sec:svd}

The Singular Value Decomposition (SVD) exists for \underline{any} matrix $\mathbf{A}\in\mathbb{C}^{m\times n}$, and can be thought of as the generalization of the eigendecomposition (cf. section \ref{sec:diagonalization}) to non-square matrices. The decomposition comes up incessantly in the context of data analysis. In general, the decomposition has the form:

\begin{equation}
\mathbf{A} = \mathbf{U\Sigma V^{\dagger}}
\end{equation}

Where $\mathbf{U^{\dagger}}\mathbf{U} = \mathbf{I}$, $\mathbf{V}$ is unitary, and $\Sigma$ is a diagonal matrix with real and positive entries $\sigma_i^2$ along the diagonal, so that $\sigma_1 \geq \sigma^2_2 \geq ...\geq \sigma^2_n$. $\mathbf{U}$ is the matrix of left singular vectors, which are the eigenvectors of $\mathbf{A A^{\dagger}}$. $\mathbf{V}$ is the matrix of right singular vectors, which are the eigenvectors of $\mathbf{A^{\dagger} A}$. 
\\

While these properties are always true, unfortunately people use a range of conventions when it comes to the size of $\mathbf{U}$, $\mathbf{\Sigma}$ and $\mathbf{V}$. \citeasnoun{friedman2001elements} uses the the convention where $\Sigma$ is square, i.e. $\mathbf{U}:\ m\times n$, $\mathbf{\Sigma}:\ n\times n$ and $\mathbf{V}:\ n\times n$. This means that $U$ does not contain the full set of $m$ left singular vectors. Note that, in this case, $\mathbf{U^{\dagger}}\mathbf{U} = \mathbf{I}$ but $\mathbf{U}\mathbf{U^{\dagger}} \neq \mathbf{I}$. \possessivecite{friedman2001elements} convention is of advantage in the context of data analysis, where the data matrix tends to be "tall and skinny" (i.e. $m>>n$), and only the first $n$ left singular vectors are relevant. Also, a letting $\Sigma$ be a square matrix significantly simplifies calculations.  
\\

The alternative is for $\mathbf{U}$ and $\mathbf{V}$ to be square, in which case they contain the full set of left and right singular vectors, and $\Sigma$ has dimension $m\times n$, with $0$ entries in rows $i>n$.