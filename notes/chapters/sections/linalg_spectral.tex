\section{Spectral Theorems, Diagonalization}

Spectral theorems deal with diagonalizable linear operators. A hermitian matrix $\mathbf{A} \in \mathbb{C}^{n\times n}$ can be expressed as:

\begin{equation}
\mathbf{A} = \left[\begin{array}{cccc}
\vrule&\vrule&\hdots&\vrule\\
v_1&v_2&\ddots&v_n\\
\vrule&\vrule&\hdots&\vrule
\end{array}\right]\left[\begin{array}{cccc}
\lambda_1&0&\hdots&0\\ 
0&\lambda_2&\hdots&0\\
\vdots&\vdots&\ddots&\vdots\\
0&\hdots&\hdots&\lambda_n
\end{array}\right]
\left[
\begin{array}{ccc}
\rule[.5ex]{3.5em}{0.4pt}&v_1&\rule[.5ex]{3.5em}{0.4pt}\\
\rule[.5ex]{3.5em}{0.4pt}&v_2&\rule[.5ex]{3.5em}{0.4pt}\\
\vdots&\ddots&\vdots\\
\rule[.5ex]{3.5em}{0.4pt}&v_n&\rule[.5ex]{3.5em}{0.4pt}\\
\end{array}
\right]
\end{equation}

Where $v_i$ are the eigenvectors and $\lambda_i$ are the eigenvalues of $\mathbf{A}$. Since $\mathbf{A}$ is Hermitian, the eigenvalues are real and positive and the eigenvectors are orthonormal. The diagonalization enables the expression of $\mathbf{A}$ in terms of projections on the eigenvectors:

\begin{equation}
\mathbf{A} = \sum_i \lambda_i (v_i \otimes v_i)
\end{equation}

Where $\otimes$ is the outer product. Since the eigenvectors are an orthonormal basis, $\sum_i v_i \otimes v_i = \mathbb{I}$.