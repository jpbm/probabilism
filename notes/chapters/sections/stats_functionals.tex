\section{Empirical Distribution Function}
It may be necessary to perform non-parametric estimation of the CDF $F$ of a set of random variables $X_1, X_2, ... , X_n \sim F$. 

The empirical distribution function $\hat{F}_n$ is the CDF that puts mass $1/n$ at each point $X_i$.

\begin{equation}
\hat{F}_n(x) = \frac{\sum_{i=1}^n I(X_i \leq x)}{n}
\end{equation}

Where $I(X_i \leq x) = \left\{\begin{array}{c} 1\ \mathrm{if\ } X_i \leq x\\ 0\ \mathrm{if\ } X_i > x \end{array} \right.$. 

The empirical CDF is discrete, even when the random variable it is based on may be continuous. 

At a given point $x$, $\hat{F}_n(x)$ is an unbiased estimator of $F(x)$. 

\begin{itemize}
\item $\mathbb{E}\hat{F}_n(x) = F(x)$
\item $\mathbb{V}\hat{F}_n(x) = 0+\mathrm{MSE} = \frac{F(x)(1-F(x))}{n}$
\item $\hat{F}_n(x) \xrightarrow{P} F(x)$
\end{itemize}

The Glivenko-Cantelli Theorem guarantees that, if $X_1,X_2,...,X_n \sim F$, then:

\begin{equation}
\sup_x |\hat{F}_n(x) - F(x)|\xrightarrow{P} 0
\end{equation}

\subsection{Confidence Measures for the Empirical CDF}
A confidence interval for the empirical CDF is given through the Dvoretzky-Kiefer-Wolfowitz (DKW) Inequality:

\begin{equation}
\mathbb{P}\left(\sup_x|F(x) - \hat{F}_n(x)|>\epsilon \right)\leq 2 e^{-2n\epsilon^2}
\end{equation}

A nonparametric $1-\alpha$ confidence band is then:

\begin{equation}
\begin{array}{c}
L(x) = \max\{\hat{F}_n -\epsilon_n, 0 \}\\
\\
U(x) = \min\{\hat{F}_n + \epsilon_n, 1\}
\end{array}
\end{equation}

with $\epsilon_n = \sqrt{\frac{1}{2n}\log\left(\frac{2}{\alpha}\right)}$.

\begin{equation}
\mathbb{P}\left(L(x) \leq F(x) \leq U(x) \right) \geq 1-\alpha
\end{equation}



\section{Statistical Functionals}
A functional is, roughly speaking, a function of a function. The fourier transform of a function is a functional. A \textit{statistical functional} is any function of the CDF, $F$. Examples are the mean, the variance, or the median. 

\subsection{Plug-in Estimator}
The plug-in estimator of $\theta = T(F)$ is given by $\hat{\theta} = T(\hat{F}_n)$. In other words, the estimate of the CDF is used instead of the true $F$, resulting in an estimator.

\subsection{Linear Functionals}
Functionals of the form $T(F) = \int r(x) dF(x)$ are linear functionals.  

\subsection{Plug-in Estimator for Linear Functionals}
\begin{equation}
T(\hat{F}_n) = \int r(x) d\hat{F}_n(x) = \frac{1}{n}\sum^{n}_{i=1}r(X_i)
\end{equation}

\subsection{Examples: Mean, Variance, Sample Variance, Sample Correlation}
\citeasnoun{wasserman2013all} pp. 100
