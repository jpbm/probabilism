\chapter{Stochastic Processes and Time Series Analysis}

\begin{multicols}{2}[\subsubsection*{Contents of this chapter}]
   \printcontents{}{1}{\setcounter{tocdepth}{2}}
\end{multicols}

In the discrete case, time series $X$ is a set of random variables $X = \{X_1, X_2, X_3,...\}$, or alternatively $X = \{X_t: t\in T\}$ where $T$ is the index set. 

\section{Branching Processes}

\section{Markov Chains}
Markov Chains are time series in which the probability of a state depends only on the state preceding it. That is:

\begin{equation}
p(X_{t+1}|X_t, X{t-1},...,X_{1},X_0) = p(X_{t+1}|X_t)
\end{equation}



\section{Martingales}
Martingales are time series in which the expected value of a state is identical to the expected value of the state preceding it. That is:

\begin{equation}
\mathbb{E}X_{t+1} = \mathbb{E}X_{t}
\end{equation}


\subsection{Martingale Convergence Theorem}

\section{Hidden Markov Models}

\section{Ito Calculus}

\section{Chaos Analysis}

\section{Noise}

\subsection{White Noise}
White noise contains all frequencies with the same intensity. That is, it has constant power spectral density. Infinite bandwidth white noise is a purely theoretical concept, but if noise has a flat power spectrum across the frequencies of interest, it is referred to as white noise. 

In discrete time, white noise is a series of uncorrelated random variables with zero mean and finite variance. When the variables all have normal distribution, the noise is referred to as Gaussian white noise. 

\subsection{Gaussian Noise}
Gaussian noise is typically understood to be white noise in which the shocks have a normal distribution.

\subsection{Brownian Noise}
Brownian noise is noise corresponding to random motion (i.e. Brownian motion). The power spectrum is proportional to $\frac{1}{f^2}$.